\section{Stock selection methodology}
\label{sec:usage:stockselection}

Given a correlation graph $G=(V,E)$, we wish to select $k$ stocks (represented 
as vertices) such that each stock is as independent as possible of the other 
stocks. Two random variables $X$ and $Y$ are independent if and only if their 
joint cumulative distribution function may be written as 
$F_{X,Y}(x,y)=F_X(x)F_Y(y)$. With this formal definition, Santos \textit{et 
al.} notes that ``we say two random variables X and Y are dependent if they
are not independent'' so the problem then becomes one of ``how to measure and 
detect dependence from the observation of the two random 
variables''\cite{santos2013}. With the correlation graph $G$, we now have 
a measure of dependence. However, it was made clear that, in general, 
it is incorrect to say that ``uncorrelated'' equates ``independence''. 
Regardless, non-correlation is still a useful tool to inform stock selection 
for portfolio management (Section~\ref{sec:intro:finance}). So, we select $k$ 
stocks such that each stock is as \textit{uncorrelated} as possible with the 
other stocks.

As $V_i$ and $V_j$ have no edge when Corr($i,j$) is below some threshold $p$, 
we might consider the following stock selection strategy:

\tablespacing
\begin{algorithm}[H]
	\caption{Naive stock selection strategy}\label{alg:usage:stockselection1}
	\begin{algorithmic}[1]
		\Procedure{}{$k$ is the number of stocks to select and 
			$A$ is the adjacency matrix of $G=(V,E)$}
		\State $z \gets$ Comb($|V|,k$)*
		\State $min \gets \infty$
		\State $index \gets 0$
		\State \textbf{loop from} $i=1$ \textbf{to} $\text{len}(z) = 
		{|V| \choose k}$:
		\State \indent $c = 0$ (the number of connections)
		\State \indent \textbf{loop from} $j = 1$ \textbf{to} $k-1$:
		\State \indent \indent \textbf{loop from} $l=j+1$ \textbf{to} $k$:
		\State \indent \indent \indent $c \gets c+A_{z_{j,i},z_{l,i}}$
		\State \indent \textbf{If} $c < min$ \textbf{then}
		\State \indent \indent $min \gets c$
		\State \indent \indent $index \gets i$
		\State \indent \textbf{Else If} $c = min$ \textbf{then}
		\State \indent \indent Randomly determine if replacing $min$ and 
		$index$ or not
		\State \textbf{return} $z_{,index}$
		\EndProcedure
	\end{algorithmic}
	*Comb($|V|,k$) is a function that returns a matrix of all possible 
	combinations of $k$ values selected from $|V|$ i.e. $z_{,i}=<z_{1,i} = 1, 
	z_{2,i} = 5, z_{3,i} = 7>$ is a combination for $|V|=10,k=3$.
\end{algorithm}
\bodyspacing

\noindent This becomes computationally difficult as $|V|$ increases and does 
not account for average stock returns; as discussed in 
Section~\ref{sec:intro:finance}, stocks with positive drift and high volatility 
are shown to improve the performance of the portfolio over time in the ``buy 
and hold'' model, or improve gains in the case of rebalancing. These two 
criteria may be used to minimize the space of stocks to select from. Drift over 
time is simplified as the average price difference between any two consecutive 
points in time. We let $p_d$ and $p_v$ denote the threshold 
values (with respect to averages) for drift and volatility respectively. These 
values will be dependent on the size of the dataset as we wish to preselect 
$p_d$ and $p_v$ such that $k<d<2k$ to reduce the computational burden. The 
selection strategy then becomes

\tablespacing
\begin{algorithm}[H]
	\caption{Adjusted stock selection strategy}\label{alg:usage:stockselection2}
	\begin{algorithmic}[1]
		\Procedure{}{$k$ is the number of stocks to select and 
			$A$ is the adjacency matrix of $G=(V,E)$. Let $D$ be the vector of 
			sample average price differences over time for each stock in $A$. 
			Let $V$ be the vector of sample average standard deviation for the 
			price of each stock in $A$.}
		
		\Function{Drift}{}
		\State $rmv = [\ ]$
		\State \textbf{loop from} $i = 1$ to $\text{len}(A)$:
		\State \indent \textbf{If} $D_i \leq \text{Avg}(D) + p_d$ \textbf{then} 
		$rmv$.append($i$)
		\State Delete all values in $rmv$ from the rows and columns of $A$
		\EndFunction
		
		\Function{Volatility}{}
		\State $rmv = [\ ]$
		\State \textbf{loop from} $i = 1$ to $\text{len}(A)$:
		\State \indent \textbf{If} $V_i \leq \text{Avg}(V) + p_v$ \textbf{then} 
		$rmv$.append($i$)
		\State Delete all values in $rmv$ from the rows and columns of $A$
		\EndFunction

		\Function{Selection}{}
		\State $z \gets$ Comb($|V|,k$)*
		\State $min \gets \infty$
		\State $index \gets 0$
		\State \textbf{loop from} $i=1$ \textbf{to} $\text{len}(z) = 
		{|V| \choose k}$:
		\State \indent $c = 0$ (the number of connections)
		\State \indent \textbf{loop from} $j = 1$ \textbf{to} $k-1$:
		\State \indent \indent \textbf{loop from} $l=j+1$ \textbf{to} $k$:
		\State \indent \indent \indent $c \gets c+A_{z_{j,i},z_{l,i}}$
		\State \indent \textbf{If} $c < min$ \textbf{then}
		\State \indent \indent $min \gets c$
		\State \indent \indent $index \gets i$
		\State \indent \textbf{Else If} $c = min$ \textbf{then}
		\State \indent \indent Randomly determine if replacing $min$ and 
		$index$ or not
		\State \textbf{return} $z_{,index}$
		\EndFunction
		\EndProcedure
	\end{algorithmic}
	*Comb($|V|,k$) is a function that returns a matrix of all possible 
	combinations of $k$ values selected from $|V|$ i.e. $z_{,i}=<z_{1,i} = 1, 
	z_{2,i} = 5, z_{3,i} = 7>$ is a combination for $|V|=10,k=3$.
\end{algorithm}
\bodyspacing