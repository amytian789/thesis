\section{Stock selection methodology}
\label{sec:usage:stockselection}

Given a correlation graph $G=(V,E)$, we wish to select $k$ stocks (represented 
as vertices) such that each stock is as independent as possible of the other 
stocks. Two random variables $X$ and $Y$ are independent if and only if their 
joint cumulative distribution function may be written as 
$F_{X,Y}(x,y)=F_X(x)F_Y(y)$. With this formal definition, Santos \textit{et 
al.} notes that ``we say two random variables X and Y are dependent if they
are not independent'' so the problem then becomes one of ``how to measure and 
detect dependence from the observation of the two random 
variables''\cite{santos2013}. With the correlation graph $G$, we now have 
a measure of dependence. However, it was made clear that, in general, 
it is incorrect to say that ``uncorrelated'' equates ``independence''. 
Regardless, non-correlation is still a useful tool to inform stock selection 
for portfolio management (Section~\ref{sec:intro:finance}). So, we select $k$ 
stocks such that each stock is as \textit{uncorrelated} as possible with the 
other stocks.

As $V_i$ and $V_j$ have no edge when Corr($i,j$) is below some threshold $p$, 
we might consider the following stock selection strategy:

\tablespacing
\begin{algorithm}[H]
	\caption{Naive stock selection strategy}\label{euclid}
	\begin{algorithmic}[1]
		\Procedure{}{$k$ is the number of stocks to select and 
			$A$ is the adjacency matrix of $G=(V,E)$}
		\State $d \gets \text{len}(A)$
		\State $z \gets$ Perm($d,k$)*
		\State $min = \infty$
		\State $index = 0$
		\State \textbf{loop from} $i=1$ \textbf{to} $d \choose k$:
		\State \indent $c = 0$
		\State \indent \textbf{loop from} $j = 1$ \textbf{to} $k$:
		\State \indent \indent \textbf{loop from} $l=1$ \textbf{to} $k$:
		\State \indent \indent \indent $c \gets c+A_{z_{i,j},z_{i,l}}$
		\State \indent \textbf{If} $c < min$ \textbf{then}
		\State \indent \indent $min = c$
		\State \indent \indent $index = i$
		\State \textbf{return} $z_{index,}$
		\EndProcedure
	\end{algorithmic}
	*Perm($d,k$) is a function that returns all possible permutations 
	of $k$ values selected from $d$ i.e. $z_{1,}=<1,5,7>$ is a permutation for 
	$d=10,k=3$.
\end{algorithm}
\bodyspacing

\noindent This becomes computationally difficult as $d$ increases and does not 
account for average stock returns; as discussed in 
Section~\ref{sec:intro:finance}, 
stocks with positive drift and high volatility are shown to improve the 
performance of the portfolio over time in the ``buy and hold'' model, or 
improve gains in the case of rebalancing. These two criteria may be used to 
minimize the space of stocks to select from. Drift over time is simplified as 
positive return averaged over time. We let $p_d$ and $p_v$ denote the threshold 
values for drift and volatility respectively. These values will be dependent on 
the size of the dataset as we wish to preselect $p_d$ and $p_v$ such that 
$k<d<2k$ to reduce the computational burden. The selection strategy then becomes

\tablespacing
\begin{algorithm}[H]
	\caption{Pruned stock selection strategy}\label{euclid}
	\begin{algorithmic}[1]
		\Procedure{}{$k$ is the number of stocks to select and 
			$A$ is the adjacency matrix of $G=(V,E)$. Let $D$ be the vector of 
			sample average returns and $V$ be the vector of sample standard 
			deviation.}
		
		\Function{Drift}{}
		\State $d \gets \text{len}(A)$
		\State $prune = [\ ]$
		\State \textbf{loop from} $i = 1$ to $d$:
		\State \indent \textbf{If} $D_i \leq p_d$ \textbf{then} 
		$prune$.append($i$)
		\State \textbf{return} prune
		\EndFunction
		
		\State \textbf{loop from} $i=1$ to len($prune$):
		\State \indent Delete row $prune_i$ and column $prune_i$ from $A$
		
		\Function{Volatility}{}
		\State $d \gets \text{len}(A)$
		\State $prune = [\ ]$
		\State \textbf{loop from} $i = 1$ to $d$:
		\State \indent \textbf{If} $V_i \leq p_v$ \textbf{then} 
		$prune$.append($i$)
		\State \textbf{return} prune
		\EndFunction
		
		\State \textbf{loop from} $i=1$ to len($prune$):
		\State \indent Delete row $prune_i$ and column $prune_i$ from $A$
		
		\Function{Selection}{}
		\State $d \gets \text{len}(A)$
		\State $z \gets$ Perm($d,k$)*
		\State $min = \infty$
		\State $index = 0$
		\State \textbf{loop from} $i=1$ \textbf{to} $d \choose k$:
		\State \indent $c = 0$
		\State \indent \textbf{loop from} $j = 1$ \textbf{to} $k$:
		\State \indent \indent \textbf{loop from} $l=1$ \textbf{to} $k$:
		\State \indent \indent \indent $c \gets c+A_{z_{i,j},z_{i,l}}$
		\State \indent \textbf{If} $c < min$ \textbf{then}
		\State \indent \indent $min = c$
		\State \indent \indent $index = i$
		\State \textbf{return} $z_{index,}$
		\EndFunction
		\EndProcedure
	\end{algorithmic}
	*Perm($d,k$) is a function that returns all possible permutations 
	of $k$ values selected from $d$ i.e. $z_{1,}=<1,5,7>$ is a permutation for 
	$d=10,k=3$.
\end{algorithm}
\bodyspacing