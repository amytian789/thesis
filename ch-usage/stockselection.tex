\section{Stock selection methodology}
\label{sec:usage:stockselection}

Given a correlation graph $G=(V,E)$, we wish to select $k < |V|$ stocks 
(represented as nodes in $G$) such that each stock is as independent as 
possible of all other stocks in the set. Two random variables $X$ and $Y$ are 
independent if and only if their 
joint cumulative distribution function may be written as 
$F_{X,Y}(x,y)=F_X(x)F_Y(y)$. With this formal definition, Santos \textit{et 
al.} notes that ``we say two random variables X and Y are dependent if they
are not independent'' so the problem may be transformed into that of ``how to 
measure and detect dependence from the observation of the two random 
variables''\cite{santos2013}. In 
general, it is incorrect to say that ``uncorrelated'' equates ``independence'', 
but correlation is still a useful tool to inform a portfolio stock selection 
procedure (see Section~\ref{sec:intro:finance}). Thus, the procedure must 
select $k$ stocks such that each stock is as \textit{uncorrelated} as 
possible with all other stocks in the set.

In a correlation graph, an existent edge indicates some form of correlation 
(visual or numerical) while a non-existent edge indicates an uncorrelated 
relationship and is a proxy for independence 
(see Section~\ref{sec:intro:correlation}). Subsequently, we might consider the 
following simple 
selection strategy that finds the set(s) of $k$ stocks with \textit{a minimum 
number of edges} between all pairs in the set:

\tablespacing
\begin{algorithm}[H]
	\caption{Simple stock selection strategy}\label{alg:usage:stockselection1}
	\begin{algorithmic}[1]
		\Procedure{}{$k$ is the number of stocks to select and 
			$A$ is the adjacency matrix of the correlation graph $G=(V,E)$}
		\State $z \gets$ Comb($|V|,k$)*
		\State $min \gets \infty$
		\State $index \gets 0$
		\State \textbf{loop from} $i=1$ \textbf{to} $\text{len}(z) = 
		{|V| \choose k}$:
		\State \indent $c \gets 0$ (the number of connections)
		\State \indent \textbf{loop from} $j = 1$ \textbf{to} $k-1$:
		\State \indent \indent \textbf{loop from} $l=j+1$ \textbf{to} $k$:
		\State \indent \indent \indent $c \gets c+A_{z_{j,i},z_{l,i}}$
		\State \indent \textbf{If} $c < min$ \textbf{then}
		\State \indent \indent $min \gets c$
		\State \indent \indent $index \gets i$
		\State \indent \textbf{Else If} $c == min$ \textbf{then}
		\State \indent \indent Randomly determine if replacing $min$ and 
		$index$ or not
		\State \textbf{return} $z_{,index}$
		\EndProcedure
	\end{algorithmic}
	*Comb($|V|,k$) is a function that returns a matrix of all possible 
	combinations of $k$ values selected from $|V|$ e.g. the vector $z_{,i}= \ 
	<z_{1,i} = 1, z_{2,i} = 5, z_{3,i} = 7>$ is a combination for $|V|=10,k=3$.
\end{algorithm}
\bodyspacing

This algorithm becomes computationally difficult as $|V|$ increases and does 
not account for other qualities of the data; as discussed in 
Section~\ref{sec:intro:finance}, stocks with positive drift and high volatility 
are shown to improve the performance of the portfolio over time in the ``buy 
and hold'' model, or improve gains in the case of rebalancing. These two 
criteria may be used to minimize the space of stocks to select from. Drift over 
time is simplified as the average price difference between any two consecutive 
points in time. We let $p_d$ and $p_v$ denote the threshold ratios
(with respect to averages) for drift and volatility respectively. These 
values will be dependent on the size of the data set as we wish to 
retroactively select $p_d$ and $p_v$ such that the number of viable stocks is 
reduced to a reasonable amount (such as 20 - 30) to reduce the computational 
burden but not reduced so much that the portfolios all end up being the same 
(which would be the case if the 
remaining number of stocks was close to $k$). The adjusted selection strategy 
may be found in Algorithm~\ref{alg:usage:stockselection2}, and the 
corresponding code may be found in 
Appendix~\ref{sec:appendicies:usage:stockselection}.

Although the revised selection procedure is still simplistic, it is certainly 
better than randomly selecting stocks and actually holds up in performance 
against a portfolio fully invested in the S\&P 500 (This result may be seen in 
Section~\ref{sec:usage:results}). Furthermore, we will see that this procedure 
also fulfills the purpose of demonstrating the viability of the VS when applied 
in a setting such as finance.

It is also important to make a note here about the creation of the numerical
correlation graphs for the eventual purpose of portfolio selection.
Because the goal is to find pairs \textit{without} edges, it is more useful to 
have a graph with \textit{more} edges so that there are less 
pairs \textit{without} edges; as such, pairs \textit{without} edges may better 
represent a more ``pure'' idea of non-correlation, making them more informative 
about the relationship between the corresponding stocks. Furthermore, having 
less pairs \textit{without} edges would make them simpler to sort through and 
interpret.
As $V_i$ and $V_j$ have \textit{no edge} when Corr($i,j$) is below some 
threshold $t$, we set a low threshold of $t = 0.15$ 
for creating the numerical correlation graphs.

\tablespacing
\begin{algorithm}[H]
	\caption{Simple stock selection strategy 
	(adjusted)}\label{alg:usage:stockselection2}
	\begin{algorithmic}[1]
		\Procedure{}{$k$ is the number of stocks to select and 
			$A$ is the adjacency matrix of correlation graph $G=(V,E)$. 
			Let $D$ be the vector of 
			sample average price differences over time for each stock in $A$. 
			Let $V$ be the vector of sample average standard deviations for the 
			price of each stock in $A$.}
		
		\Function{Drift}{}
		\State $rmv = [\ ]$
		\State \textbf{loop from} $i = 1$ to $\text{len}(A)$:
		\State \indent \textbf{If} $D_i \leq \text{Avg}(D) * p_d$ 
		\textbf{then} 
		$rmv$.append($i$)
		\State Delete all indices in $rmv$ from the rows and columns of $A$
		\EndFunction
		
		\Function{Volatility}{}
		\State $rmv = [\ ]$
		\State \textbf{loop from} $i = 1$ to $\text{len}(A)$:
		\State \indent \textbf{If} $V_i \leq \text{Avg}(V) * p_v$ 
		\textbf{then} 
		$rmv$.append($i$)
		\State Delete all indices in $rmv$ from the rows and columns of $A$
		\EndFunction

		\Function{Selection}{}
		\State $z \gets$ Comb($|V|,k$)*
		\State $min \gets \infty$
		\State $index \gets 0$
		\State \textbf{loop from} $i=1$ \textbf{to} $\text{len}(z) = 
		{|V| \choose k}$:
		\State \indent $c \gets 0$ (the number of connections)
		\State \indent \textbf{loop from} $j = 1$ \textbf{to} $k-1$:
		\State \indent \indent \textbf{loop from} $l=j+1$ \textbf{to} $k$:
		\State \indent \indent \indent $c \gets c+A_{z_{j,i},z_{l,i}}$
		\State \indent \textbf{If} $c < min$ \textbf{then}
		\State \indent \indent $min \gets c$
		\State \indent \indent $index \gets i$
		\State \indent \textbf{Else If} $c == min$ \textbf{then}
		\State \indent \indent Randomly determine if replacing $min$ and 
		$index$ or not
		\State \textbf{return} $z_{,index}$
		\EndFunction
		\EndProcedure
	\end{algorithmic}
	*Comb($|V|,k$) is a function that returns a matrix of all possible 
	combinations of $k$ values selected from $|V|$ e.g. the vector $z_{,i}= \ 
	<z_{1,i} = 1, z_{2,i} = 5, z_{3,i} = 7>$ is a combination for $|V|=10,k=3$.
\end{algorithm}
\bodyspacing