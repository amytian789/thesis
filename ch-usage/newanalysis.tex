\section{Application of the visualization system}
\label{sec:usage:newanalysis}

Let $X$ be a $n\times d$ data matrix where there are $n$ observations of $d$ 
variables. The visualization system should be applied as follows:

\tablespacing
\begin{enumerate}
	\item Create four different numerical correlation graphs 
	$G_i^{\text{num}}=(V,E)$ for all $i\in \{1,...,4\}$, one for each 
	correlation coefficient described in Section~\ref{sec:intro:correlation}.
	
	\item Run the VS on the same dataset. Stage 1 of the system utilizes the 
	active learning algorithm selected in Chapter~\ref{ch:al} to learn what is 
	``correlated'' and ``not correlated.'' Stage 2 of the system then iterates 
	through all possible unlabeled plot pairs and returns a visual graph 
	$G=(V,E)$ where 

\begin{algorithm}
	For $i,j\in \{1,...,d\}$, $E_{i,j}$ exists if the plot of $j$ against $i$ 
	is ``correlated''
\end{algorithm}

	As a reminder, correlation is used loosely to refer to a visual 
	interpretation of the mathematical term.

	\item Utilize the graph comparison methodology explained in 
	Chapter~\ref{ch:gc} to find the difference of each pair 
	$(G,G_i^{\text{num}})$ for all $i \in \{1,...,4\}$. Select numerical 
	graph $G_i^{\text{num}}$ such that $\text{diff}(G,G_i^{\text{num}}) 
	\leq \text{diff}(G,G_j^{\text{num}})$ for all $j\neq i$. The 
	numerical correlation graph $G_i^{\text{num}}$ is then the graph which 
	best matches the visual interpretation of the relationships among the 
	dataset's variables.
\end{enumerate}
\bodyspacing

Because the question we are interested in answering goes beyond simply finding 
the ``best'' numerical correlation graph, we will go further and use the 
``best'' resulting numerical graph to select stocks.