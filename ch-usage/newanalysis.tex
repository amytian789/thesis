\section{Application of the visualization system}
\label{sec:usage:newanalysis}

Now the issue is that there are many correlation graphs which may be 
computed with different correlation coefficients; given the large dataset, it 
is unclear which metric might be the best one. This is where the visualization 
system comes in and ties everything together. By learning the user's interests 
and automatically labeling correlated vs uncorrelated pairs of variables, the 
VS outputs a visual correlation graph that may be used as a ``sanity check'' 
against the various numerical correlation graphs and provide a guide for 
selecting the most intuitive numerical correlation graph.

Let $X$ be a $n\times d$ data matrix where there are $n$ observations of $d$ 
variables. The final procedure is as follows:

\tablespacing
\begin{enumerate}
	\item Create four different numerical correlation graphs 
	$G^i=(V,E)$ where $|V| = d$ for all $i\in \{1,...,4\}$ with 
	threshold $t = 0.15$, one for each correlation coefficient described in 
	Section~\ref{sec:intro:correlation} (Pearson's, Spearman's, Kendall's, and 
	distance correlation).
	
	\item Run the VS on the same dataset. Stage 1 of the system utilizes the 
	active learning algorithm selected in Chapter~\ref{ch:al} to learn what is 
	``correlated'' and ``not correlated.'' Stage 2 of the system then iterates 
	through all possible unlabeled plot pairs and returns a visual graph 
	$G=(V,E)$ where 

\begin{algorithm}
	For $i,j\in \{1,...,d\}$, $E_{i,j}$ exists if the plot of $j$ against $i$ 
	is ``correlated''
\end{algorithm}

	As a reminder, correlation is used loosely to refer to a visual 
	interpretation of the mathematical term.

	\item Utilize the graph comparison methodology explained in 
	Chapter~\ref{ch:gc} to find the difference of each pair 
	$(G,G^i)$ for all $i \in \{1,...,4\}$. Select numerical graph $G^*$ such 
	that $\text{diff}(G,G^*) \leq \text{diff}(G,G^i)$ for all $i\in 
	\{1,...,4\}$. The numerical correlation graph $G^*$ is then the graph which 
	best matches the visual interpretation of the relationships among the 
	dataset's variables.
	
	\item Using the stock selection procedure in 
	Section~\ref{sec:usage:stockselection}, create a portfolio $P^i$ from each 
	graph $G^i$ (including portfolio $P^*$ from the graph selected as $G^*$). 
	
	\item Observe the performance of each portfolio until 2014.
\end{enumerate}
\bodyspacing

Normally, it would be standard to simply use the selection procedure with $G^*$ 
as an input to determine the desired ``buy and hold'' portfolio, record the 
yearly returns, and then be done. However, because the purpose of this chapter 
is to demonstrate the usage and viability of the VS in aiding with the selectio 
of an appropriate numerical procedure (given a dataset and some desired 
purpose), it is necessary to compare the results of all options among one 
another. 
It is natural to expect that the portfolio $P^*$ will perform the best 
as that is the purpose of the visualization system, after all. 
By comparing the results of $P^*$ against the other numerical options, we may 
determine just how well the system captured the user's intuition regarding 
independence (which is not as straightforward as a simple metric).
It should also be noted that, while a rebalancing method performs better in 
practice~\cite{liuh2016}, the ``buy and hold'' is better-suited for observing 
the portfolio performances over time, providing a more concrete basis of 
comparison for the selected correlation coefficient. The final code for the 
application procedure may be found in 
Appendix~\ref{sec:appendicies:usage:newanalysis}.