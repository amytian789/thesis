\section{Summary}
\label{sec:intro:summary}

%%%
% copied & pasted from "solution.tex" 
In this work, we tackle the problem by developing a sophisticated visualization
system (abbreviated VS) to explore the data and numerical model in a different
way. Specifically, we focus on two aspects of the VS which address the two
problems raised in Section~\ref{sec:intro:problem}: (1) the procedure of sorting
through plots is efficiently automated by learning the user's interests, and (2)
the procedure of comparing the numerical and visual output is also automated.
This allows the system to find visually interesting relationships that the
numerical model may have missed and/or toss out relationships which turn out to
be uninteresting. Furthermore, this allows future analysts to combine visual
feedback with the numerical feedback from estimators to make better decisions
during data analysis and provide clear justification of their decisions.
%%%

A broad overview of the VS and its framework may be found in
Chapter~\ref{ch:visualizer}. Chapter~\ref{ch:al} then focuses on the active
learning stage of the VS, which is part of the first solution to the problems
raised in Section~\ref{sec:intro:problem}. The chapter details and simulates
various active learning methods to be used in the financial application in
Chapter~\ref{ch:usage}. Chapter~\ref{ch:gc} focuses on the second solution to
the problems raised in Section~\ref{sec:intro:problem}. The chapter is concerned
with the VS output, which quantifies the differences between numerical and
visual correlation graphs for the user. Subsequently, the chapter details
various graph comparison methods and ultimately selects one for usage in the
financial application in Chapter~\ref{ch:usage}. Finally,
Chapter~\ref{ch:conclusion} recaps the work and presents future extensions of
the VS.

