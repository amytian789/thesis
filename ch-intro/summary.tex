\section{Road map}
\label{sec:intro:summary}

In this thesis, we tackle the problem of sorting through all possible pairwise
scatter plots in high dimensions by developing a sophisticated visualization
system (abbreviated VS) to intelligently and automatically explore the data. 
Specifically, we focus on two aspects of the VS: 
(1) efficiently automating the procedure of sorting through plots by learning 
the user's interests, and (2) the procedure of comparing the numerical and 
visual output (in the form of correlation graphs).
This allows the system to find visually interesting relationships that the
numerical model may have missed and/or toss out relationships which turn out to
be uninteresting. Furthermore, this allows future analysts to combine visual
feedback with the numerical feedback from estimators to make better decisions
during data analysis and provide clear justification of their decisions.
%%%

A broad overview of the VS and its framework may be found in
Chapter~\ref{ch:visualizer}. Chapter~\ref{ch:al} focuses on the active
learning stage of the VS, which answers the problem of high-dimensional 
visualization raised in Section~\ref{sec:intro:problem}. 
The chapter details and simulates various active learning methods to be used in 
the financial application in Chapter~\ref{ch:usage}. 
Chapter~\ref{ch:gc} focuses on the application of the visual output in relation 
to the financial application of Chapter~\ref{ch:usage}. Specifically, 
Chapter~\ref{ch:gc} is concerned with the VS output in the context of graphs, 
and it discusses methods to 
quantify the differences between numerical and visual graphs. Furthermore, the 
chapter proposes a procedure to find the numerical graph most similar to (least 
different from) the visual graph. Though our  application is concerned with 
correlation graphs, the methods discussed in Chapter~\ref{ch:gc} may apply to 
any other sort of graph (e.g. a graphical model). 
Chapter~\ref{ch:usage} contains the financial application where the 
visual correlation graph output from the VS is used to aid in the selection of 
the most suitable numerical correlation graph for portfolio selection with our 
particular data set. The chapter further proposes a simple but effective 
portfolio selection procedure given the numerical correlation graph.
Finally, Chapter~\ref{ch:conclusion} recaps the thesis and presents further 
extensions.

