\section{Correlation graphs}
\label{sec:intro:correlation}

Correlation graphs are one way to discover the dependency structure among different stocks whose returns may be represented as random variables following some distribution. Let $G=(V,E)$ be an undirected graph with vertices $V_{1},...,V_{d}$ (a $d$-dimensional distribution) and edges $E_{i,j}\in\{0,1\}$. We set $E_{i,j}=1$ when there is an edge between $V_i$ and $V_j$, and 0 otherwise. An edge is drawn between $V_i$ and $V_j$ iff the two random variables are correlated. This graph can be drawn from a correlation matrix $\sum$ where $\sum_{i,j}=corr(V_1,V_2)$ with the following heuristic:\\

\begin{algorithm}
	If $\sum_{i,j}>p$, draw edge $E_{i,j}$ where $p$ is the $p$-value for the desired confidence level
\end{algorithm}

We differentiate between ``visual correlation'' (which can be thought more of as ``pairs of variables that marginally appear dependent'') and the more common mathematical interpretation of ``correlation''. More specifically, we would like to visually understand what a correlation graph looks like and compare it to correlation graphs constructed with the traditional interpretation of correlation. What follows is an overview of common numerical methods to estimate the correlation between two random variables.

\subsection{Pearson correlation}

\subsection{Kendall's tau}

\subsection{Spearman correlation}