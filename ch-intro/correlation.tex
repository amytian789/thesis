\section{Correlation graphs}
\label{sec:intro:correlation}

Independence graphs are one way to discover the dependency structure among
different stocks whose prices may be represented as random variables following
some distribution. The importance of independence among assets is explained 
later in Section~\ref{sec:intro:finance}.

Let $G=(V,E)$ be an undirected graph formed from the data set 
$X$ with $d$ observations of $n$ variables. 
$G$ has vertices $|V| = n$ and edges $E_{i,j}\in\{0,1\}$. 
$E_{i,j}=1$ when there is an edge between $V_i$ and $V_j$, and 0 otherwise. 
Edges are determined with the following heuristic: \\

\begin{algorithm}
	$E_{i,j} = 0$ for $i\neq j$ if and only if $X_i \perp X_j$ (the two 
	variables are independent)
\end{algorithm}

Unfortunately, independence graphs are difficult to construct in practice since 
independence is determined by the true joint distributions of all $n$ variables 
in $X$, which is difficult to determine. However, the independence graph 
$G$ may be estimated by the sample correlation graph 
$\hat{G}=(V,E)$. While it is true that non-correlation does 
not always imply independence, it is a useful estimate for \textit{some} form 
of independence between two variables. 
It is important to be careful about the terminology usage. We use
``correlation'' to refer to \textit{any form} of pair-wise dependence. 

There are two main methods to estimate the dependence between two random 
variables; the first and most common is \textbf{numerical correlation}. 
By ``numerical correlation'', we do not mean the traditional mathematical 
interpretation of ``Pearson's correlation'' but, instead, anything which 
involves explicitly computing an estimate of the correlation e.g. Pearson's, 
Spearman's, and various other correlation metrics which are described later in 
this section. 
Let $\hat{G}^{\text{num}}$ be the numerical correlation graph of $X$. 
$\hat{G}^{\text{num}}$ can be drawn from a matrix $P$ where 
$P_{i,j}=p(X_i,X_j)$ (the $p$-value of some sample correlation coefficient 
computed from the variables $X_i$ and $X_j$) with the following heuristic:\\

\begin{algorithm}
	$E_{i,j} = 1$ for $i\neq j$ if $P_{i,j} \leq p$ where $p$ is the $p$-value 
	for the desired confidence level
\end{algorithm}

Alternatively and more simplistically, the numerical correlation graph can be 
drawn from a correlation matrix $\hat{\Sigma}$ where 
$\hat{\Sigma}_{i,j}=\hat{\rho}(X_i,X_j)$ (some sample correlation coefficient 
computed from $X_i$ and $X_j$) with the following heuristic:\\

\begin{algorithm}
	$E_{i,j} = 1$ for $i\neq j$ if $|\hat{\Sigma}_{i,j}| \geq \lambda$ where 
	$0 \leq \lambda \leq 1$ is the desired threshold value for a 
	``significant'' correlation (e.g. $\lambda = 0.75$)
\end{algorithm}

\noindent We implement this latter approach when drawing correlation 
graphs for the financial application in Chapter~\ref{ch:usage}.

The second method is \textbf{visual correlation}. By ``visual correlation'', we 
mean ``pairs of variables that marginally appear dependent''. 
Subsequently, the visual graph $\hat{G}^{\text{vis}}$ of $X$ can be drawn with 
the following heuristic:\\

\begin{algorithm}
	$E_{i,j} = 1$ for $i\neq j$ if the pairwise scatter plot of $(X_i, X_j)$ 
	contains a distinct trend
\end{algorithm}

Visual correlation is, admittedly, subjective by nature 
and, as a consequence, slightly vague. Traditionally, 
numerical correlation is used over visual correlation due to its objectivity. 
However, visual correlation is still necessary in conjunction with 
numerical correlation. Consider if the joint distributions of the variables 
changed over time or if there were outliers as in the example of 
Section~\ref{sec:intro:problem}. While the dependency of the variables could 
not be captured by standard correlation metrics, we have a notion of the visual 
correlation between them. But because visual interpretations are subjective and 
dependent on how the data is plotted, numerical correlation is still useful. 
The two views on estimating the population correlation do not have to be 
mutually exclusive and actually supplement each other's faults instead.
To capture both qualities, a visual correlation graph may be utilized to check 
the suitability of various numerical correlation metrics that can be applied to 
the data.
What follows is an overview of common numerical methods 
to estimate the correlation between two random variables. It will become clear 
that the numerical correlation metrics below are sensitive in some way or 
another, further emphasizing the need for both methodologies.

\subsection{Pearson's correlation}

Pearson's correlation measures the linear dependence among two random variables
$X$ and $Y$. In a population, the Pearson correlation is given by:
$$\rho_{X,Y}=\frac{\Cov(X,Y)}{\sigma_X \sigma_Y} =
\frac{\E(XY)-\E(X)\E(Y)}{\sqrt{\E(X^2)-\E(X)^2} \sqrt{\E(Y^2)-\E(Y)^2}}$$ 

As alluded earlier, the quantity above is typically unknown in practice since 
the expectation requires knowing the true joint distribution of $X$ and $Y$. 
Instead, the population correlation is typically estimated by the sample 
correlation. Given $n$ observations of $X$ and $Y$, the sample expectation is 
given by the formula  
$\E(X)=\frac{1}{n}\sum\limits_{i=1}^{n}x_i$. 
By substituting into the formulation above and multiplying by $n^2/n^2$, we can 
estimate the sample Pearson correlation with the following:
$$\hat{\rho}_{x,y}^\text{Pear}=
\frac{n \sum\limits_{i=1}^{n} x_i y_i - \sum\limits_{i=1}^{n} x_i
	\sum\limits_{i=1}^{n} y_i}
{\sqrt{n\sum\limits_{i=1}^{n} x_i^2-\left(\sum\limits_{i=1}^{n} x_i\right)^2} 
	\sqrt{n\sum\limits_{i=1}^{n} y_i^2-\left(\sum\limits_{i=1}^{n} y_i\right)^2}}$$ 

\noindent such that $-1 \leq \hat{\rho}_{x,y}^\text{Pear} \leq 1$. With perfect 
positive and negative linear 
dependence respectively, $\hat{\rho}_{x,y}^\text{Pear}=\pm1$.
It is important to note that $\hat{\rho}_{x,y}^\text{Pear}=0$ does not 
necessarily indicate independence, though it is an indication of 
\textit{linear} independence.

\subsection{Spearman's correlation}

Spearman's correlation is more broad than Pearson's; it measures monotonic
dependence among two random variables $X$ and $Y$. Monotonic functions are
either strictly increasing or decreasing; while linear functions are monotonic,
monotonic functions are not necessarily linear. Subsequently, Spearman's
correlation may also capture non-linear dependencies. Spearman's correlation is
computed by calculating the Pearson's correlation among ``ranked variables''. 
Each sample observation $x_i$ of $X$ is ranked from 1 to $n$ based on its 
position relative to $x_j, j\in\{1,...,n\}\backslash{i}$. The ranking is also 
computed for all observations of $Y$. Then the difference of a sample 
$(x_i,y_i)$ is defined as $d_i=x_i-y_i$, and the sample Spearman's correlation 
is computed as:
$$\hat{\rho}_{x,y}^{\text{Spear}}=
1-\frac{6 \sum\limits_{i=1}^{n}d_i^2}{n(n^2-1)}$$

\noindent such that $-1 \leq \hat{\rho}_{x,y}^{\text{Spear}} \leq 1$. With 
perfect increasing and decreasing monotonic dependence respectively,
$\hat{\rho}_{x,y}^{\text{Spear}} = \pm1$. Again, it is important to note that 
$\hat{\rho}_{x,y}^{\text{Spear}}=0$ does not necessarily indicate 
independence, though it is an indication of \textit{monotonic} independence.

\subsection{Kendall's tau}

Similar to Spearman's correlation, Kendall's tau is another method of
identifying monotonic dependence among two random $X$ and $Y$ as it also
computes correlation among ranked variables. However, it does not utilize the
difference among a single sample. Instead, it compares pairs of samples among
each other. For $i\not=j$, $(x_i,y_i)$ and $(x_j,y_j)$ are defined as
``concordant'' if the ranks of both elements agree i.e. $x_i > x_j$ and $y_i >
y_j$ or $x_i < x_j$ and $y_i < y_j$. Pairs are defined as ``discordant'' if the
ranks of both elements disagree i.e. $x_i > x_j$ and $y_i < y_j$ or $x_i < x_j$
and $y_i > y_j$. In the case where ranks of either element are equal, the pair
is ignored. Let $c$ denote the number of concordant pairs and $d$ denote the 
number of discordant pairs. Then the sample Kendall's tau is computed as:
$$\hat{\rho}_{x,y}^{\text{Kend}}=\frac{c-d}{n(n-1)/2}$$

\noindent such that $-1 \leq \hat{\rho}_{x,y}^{\text{Kend}} \leq 1$. 
Kendall's tau is less sensitive to errors in the data as its correlation is
based on pairs of samples rather than deviations within a single sample, though 
the resulting coefficients have the same interpretations. As with Spearman's
correlation, $\hat{\rho}_{x,y}^{\text{Kend}}=\pm1$ with perfect increasing 
and decreasing monotonic dependence respectively. Furthermore, 
$\hat{\rho}_{x,y}^{\text{Kend}}=0$ does not necessarily indicate
independence, though it is an indication of \textit{monotonic} independence.

\subsection{Distance correlation}

While the aforementioned correlation metrics are well-known and commonly used, 
they are constrained to monotonic functions. Distance correlation was first 
proposed in 2007 as a way to further test for non-monotonic dependence between 
$X$ and $Y$~\cite{szekely2007}. The distance correlation is a function of the 
distance covariance and distance variance of $X$ and $Y$. The $n\times n$ 
matrices $a$ and $b$ denote the distance matrices of $X$ and $Y$, 
respectively. The elements $a_{k,l}$ and $b_{k,l}$ are respectively defined as 
$||X_k-X_l||$ and $||Y_k-Y_l||$ for all $k,l=1,2,...,n$ where $||z||$ is the 
Euclidean norm $\sqrt{z_1^2+...+z_n^2}$. Then the sample distance 
covariance and sample distance are given by:

\tablespacing
\begin{itemize}
	\item $\mathrm{dCov}(X,Y) = \sqrt{\frac{1}{n^2} \sum\limits_{k=1}^{n} 
	\sum\limits_{l=1}^{n} A_{k,l} B_{k,l}}$
	\item $\mathrm{dVar}(X) = \mathrm{dCov}(X,Y)$
	\item $\mathrm{dVar}(Y) = \mathrm{dCov}(Y,Y)$
\end{itemize}
\bodyspacing

\noindent where $A_{k,l}=a_{k,l}-\bar{a}_{k,}-\bar{a}_{,l}+\bar{a}$ and 
$\bar{a}_{k,}$ is the $k$th row mean of $a$, $\bar{a}_{,l}$ is the $l$th column 
mean of $a$, and $\bar{a}$ is grand mean of $a$. 
Similarly, $B_{k,l}=b_{k,l}-\bar{b}_{k,}-\bar{b}_{,l}+\bar{b}$ and 
$\bar{b}_{k,}$ is the $k$th row mean of $b$, $\bar{b}_{,l}$ is the $l$th column 
mean of $b$, and $\bar{b}$ is grand mean of $b$. The sample distance 
correlation is then computed as:
$$\hat{\rho}_{x,y}^{\text{Dist}}=\frac{\mathrm{dCov}(X,Y)}
{\sqrt{\mathrm{dVar}(X)\mathrm{dVar}(Y)}}$$

\noindent 
such that $0 \leq \hat{\rho}_{x,y}^{\text{(Dist)}} \leq 1$. As before, 
$\hat{\rho}_{x,y}^{\text{(Dist)}}=1$ indicates perfect dependence. 
Like the other correlation metrics, $\hat{\rho}_{x,y}^{\text{Dist}}=0$ implies 
\textit{some} form of independence. 
However, the interpretation of $\hat{\rho}_{x,y}^{\text{Dist}}=0$ is
interesting in the sense that its corresponding population correlation metric
$\rho_{x,y}^{\text{Dist}}$ has two important properties not seen in the other 
population metrics~\cite{szekely2007}: 

\tablespacing
\begin{enumerate}
	\item $X$ and $Y$ may be of different dimensions.
	\item$\rho_{x,y}^{\text{Dist}}=0$ if and only if $X$ and $Y$ are 
	independent.
\end{enumerate}
\bodyspacing


