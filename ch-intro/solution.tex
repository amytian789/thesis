\section{Solution overview}
\label{sec:intro:solution}

We focus on the specific problem of data visualization within the framework of ``clean analysis.'' This is a specific element of the iterative, decision-making process of data analysis that provides guidelines and accountability for analysts. Specifically, we focus on developing a visualization tool to supplement in model selection after numerical methods have been applied. Model selection addresses the problem of determining which explanatory variables (commonly seen as columns of the matrix X) are informative to explain the variation in the response variable (commonly seen as the vector Y). This is related to the concept of ``sparsity,'' a statistical term referring to the fact that many coefficients of a fitted model should be 0. Model selection with sparsity aids in the interpretability of the model since there are fewer variables for data analysts to understand. The visualization system aids in this process by checking the user's concept of an ``interesting correlation'' against the supplied numerical model. This allows the system to find visually interesting relationships that the numerical model may have missed and/or toss out relationships which turn out to be uninteresting.

While scatterplots and histograms are commonly used as a preliminary tool to explore datasets and verify the effectiveness of the fitted model, their effectiveness is lost on multivariate datasets with many variables. S. Liu et al. notes that ``physical limitations of display devices and our [human's] visual system prevent the direct display and instantaneous recognition of structures with higher dimensions than two or three''~\cite{lius2016}. This is troubling as high-dimensional datasets are found in numerous fields outside of finance. One solution is to manually plot each explanatory variable against the response variable, but this becomes computationally tedious and unfeasible to sort through when there are even a few hundred variables. This problem gets even more complicated when considering various transformations or combinations of explanatory variables (``interaction terms''). Due to its tediousness, some analysts may choose to utilize numerical methods alone, but they are left without a concrete way to verify their results. Although methods for dimension reduction have been developed~\cite{lius2016}, it is still unclear how the analyst can easily check the resulting model to ensure that the variables which were culled in the dimension reduction process are actually undesirable. Without considering these possibilities, important insight might be lost and the resulting model might be unsatisfactory unbeknownst to the analyst. 

Regardless of whether high dimensional data visualization methods are computationally heavy or interaction heavy, user interactivity is still a vital component for processing high dimensions for visualization; it is simply a question of what degree~\cite{lius2016}. We develop a system that first learns what visual patterns the data analyst finds promising, querying the user where the decision tree is ambiguous. It then automatically iterates through thousands of possible plots. Finally, it suggests relationships to exclude or include, which it believes to match the data analyst's interests, and their corresponding plots. This allows users to compare and contrast visual feedback with numerical algorithms for improved model selection.
