\section{Portfolio management}
\label{sec:intro:finance}

An important application of correlation graphs is in modeling the dependencies 
among financial equities. Determining the relationship among various stocks is 
especially useful when managing portfolios. One such methodology is the ``buy 
and hold'' tactic where an investor selects a portfolio of stocks and never 
rebalances. The idea is to select diversified stocks with low correlation and 
positive drift such that losses are offset by gains, and the portfolio gains on 
average. This strategy is especially useful when transaction costs are high as 
fees are prohibitive to rebalancing gains. 

In a world with low transaction costs, however, there is more to be gained 
(less to be lost, alternatively) by frequently rebalancing the portfolio rather 
than holding. The concept for stock picking is similar to that of ``buy and 
hold'', though there are some key difference. Another component for successful 
rebalancing gains is for the stock returns to be relatively 
independent~\cite{liuh2016}. Thus, when one stock goes down, the others do not 
fall with it; with perfect independence, rebalancing gains become a function of 
the volatility of the stocks rather than of stock returns. This is further 
important because frequent asset trading may move the market and lead to 
unexpected price swings. With independent stocks, the price risk associated 
with rebalancing is minimized. In summary, rebalancing gains are best suited to 
markets with low correlation, independent returns, and high 
volatility~\cite{liuh2016}.  

Another financial application is the creation of a predictive stock model. 
Predictive models are a form of ``model selection'' where dependence rather 
than independence is important. Model selection addresses the problem of 
determining which explanatory variables (commonly seen as columns of the matrix 
$X$) are informative to explain the variation in the response variable 
(commonly seen as the vector $Y$). This is also related to the concept of 
``sparsity,'' a statistical term referring to the fact that many coefficients 
of a fitted model should be 0. Model selection with sparsity aids in the 
interpretability of the model since there are fewer variables for data analysts 
to understand. This is another way to view rebalancing; the predictive model, 
if it is to be believed, can signal price swings to come which open up 
arbitrage opportunities. Furthermore, stocks which are uncorrelated are those 
with insignificant coefficients in the resulting model; this ``negative space'' 
point of view is partly utilized in the stock selection methodology for 
correlation graphs proposed in  Section~\ref{sec:usage:stockselection}.

It must be noted that correlation alone cannot capture the full complexity of 
these financial applications; in fact, cases such as that of 
Figure~\ref{fig:intro:meplot} (right) reflect the limitations of correlation 
coefficients and reinforce the importance of visualization. An unassuming 
analyst might select the data as their correlation is not significantly 
different from zero (Table~\ref{tab:intro:me}, Dataset 2), but it turns out 
that they have selected a stocks which we can see are highly correlated!
However, it is still an important component of 
portfolio management in theory and in practice. Regardless, correlation graphs 
and their financial applications are not solutions to the two problems proposed 
in Section~\ref{sec:intro:problem}. Rather, correlation graphs and their 
financial applications are a concrete application of whatever the proposed 
solution is (A detailed application may be found in Chapter~\ref{ch:usage}, and 
a stock selection methodology for correlation graphs is proposed in 
Section~\ref{sec:usage:stockselection}). What follows is a roadmap of the 
high-dimensional visualization solution that is developed further in this 
paper. 