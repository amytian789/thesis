\section{Portfolio management}
\label{sec:intro:finance}

An important application of correlation graphs is in modeling the dependencies 
among equities. Determining the relationship among various stocks is 
especially useful when managing portfolios. One such methodology is the ``buy 
and hold'' tactic where an investor selects a portfolio of stocks and never 
rebalances. The idea is to select diversified stocks with low correlation and 
positive drift such that losses are offset by gains, and the portfolio gains on 
average. As such, portfolio selection is the main decision-making component of 
the ``buy and hold'' method. This strategy is especially useful when 
transaction costs are high as fees are prohibitive to rebalancing gains. 

In a world with low transaction costs, however, there is more to be gained 
(less to be lost, alternatively) by frequently rebalancing the portfolio rather 
than holding. The concept for rebalancing is similar to that of ``buy and 
hold'', though there are some key differences. Another component for successful 
rebalancing gains is for the stock returns to be relatively 
independent~\cite{liuh2016}. Thus, when one stock goes down, the others do not 
fall with it; with perfect independence, rebalancing gains become a function of 
the volatility of the stocks rather than of stock returns. Independence is 
further important because frequent asset trading may move the market and lead 
to unexpected price swings. With independent stocks, the price risk associated 
with rebalancing is minimized. In summary, rebalancing gains are best suited to 
markets with low correlation, independent returns, and high 
volatility~\cite{liuh2016}. 

In this thesis, we focus on the problem of portfolio selection and, 
specifically, its application to the ``buy and hold'' strategy. However, 
portfolio selection is still a component of the rebalancing strategy, so our 
findings and proposed procedures may be further extended to portfolio 
management techniques.
A detailed application may be found in Chapter~\ref{ch:usage} where
a stock selection methodology for correlation graphs is also proposed.

We acknowledge that non-correlation does not necessitate independence, but it 
is still a useful proxy for independence.
Again, it must be noted that numerical correlation alone cannot capture the 
full complexity of these financial applications; in fact, cases such as that of 
Figure~\ref{fig:intro:meplot} (right) reflect the limitations of correlation 
coefficients and reinforce the importance of visualization in aiding with the 
selection of numerical metrics. An unassuming 
analyst might select the stocks (variables) from Table~\ref{tab:intro:me}, 
Dataset 2, because their numerical correlation is not significantly different 
from zero. But, it turns out that the stocks are highly correlated, and this is 
only clear when looking for visual correlation. However, numerical correlation 
is still an important component of portfolio management in theory and in 
practice. 
As such, an important goal of this thesis is to not only develop a 
solution to the problem of high-dimensional visualization described in 
Section~\ref{sec:intro:problem} but to also consider the reason we are 
interested in visualization in the first place. As noted earlier, plotting is 
an incredibly useful tool to check numerical methods. Thus, we will also 
develop a method to quickly check the output from numerical methods with an 
analyst's own concepts of visual dependence, allowing the analyst to select the 
numerical method most suited to their application (numerical correlation graphs 
and portfolio selection, in this case). 
What follows is a road map of the high-dimensional visualization solution that 
is developed further in this thesis with 
Chapters~\ref{ch:visualizer},~\ref{ch:al}, and~\ref{ch:gc}.