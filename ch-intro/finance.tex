\section{Portfolio management}
\label{sec:intro:finance}

An important application of our system is in dependencies among financial equities. Determining the relationship among various stocks is especially useful when managing portfolios. One such consideration is a ``buy and hold'' tactic. This strategy is especially useful when transaction costs are high. Building a portfolio in such an environment means taking hedging into account so as to minimize the impact of bears on portfolio returns. Negatively correlated stocks hedge each other; as one goes down, the other goes up; in an ideal world, this results in zero net loss. The concept of sparsity is important once again. The analyst is looking to find negatively correlated instruments, but if the model is improperly selected, there may be too many connected variables. This makes interpretability and, ultimately, selection difficult as portfolio construction is performed with a fixed amount of capital.

In a world with low transaction costs, however, there is more to be gained (less to be lost, alternatively) by frequently rebalancing the portfolio rather than holding. A way to hedge the portfolio when rebalancing would be to select relatively independent assets; with perfect independence, rebalancing gains become a function of the volatility of the stocks rather than prices or other factors~\cite{liuh2016}. This is extremely important because putting in a trade for buy or sell an asset can move the market, leading to unexpected price swings. By selecting independent stocks, the analyst is able to minimize the price risk associated with rebalancing. Identification of independent stocks is another form of model selection; the goal is to select the proper model such that relationships (and in the case of stocks, the independence) among variables are easily seen. While a numerical model can certainly be fitted, it is still important to have a visualization tool to validate the results or risk losing millions of dollars.

Another financial application is the creation of a predictive model for a certain company's stock. This case is more in line with the definition of model selection noted earlier as the analyst is looking for dependence rather than independence. The analyst would like to filter out the independent assets $X$ only to leave those which can explain the response in stock $Y$. This correlation can be positive or negative, and it can be a useful aid in trading as the predictive model, if it is to be believed, can signal price swings to come which opens up arbitrage opportunities. All of these methodologies can be boiled down to this one problem: determining the relationship among different stocks. This is not a trivial task, and there are a multitude of existing numerical approaches to quantify the relationships among different assets. 