%The importance of visualization.

The proliferation of high-dimensional datasets has led analysts to rely on 
unverifiable numerical estimators. We are interested in visualizing 
high-dimensional correlation 
graphs as a way to verify numerical tests of dependence. This is important in 
asset management as it is desirable to select a portfolio of stocks that are as 
independent as possible. High-dimensional visualization is problematic because 
(1) there are too many potential plots to sort through manually (To be 
specific, there are $d\choose 2$  plots where $d$ is the number of variables 
e.g. stocks that we are interested in), which also means that (2) it is tedious 
to verify numerical results with visual results and vice versa 
(Section~\ref{sec:intro:problem}). In this work, we 
presented a visualization tool that actively learns user preferences, applies 
the fitted classifier to unlabeled data, provides visualization tools for the 
active learning output (a visual graph $G=(V,E)$), and outputs the difference 
among the numerical graph $G^{\text{num}}=(V,E)$ and $G$ 
(Chapter~\ref{ch:visualizer}).

As a specific response to the two aforementioned problems, we focused on the 
active learning and graph comparison components of the visualizer systems. We 
discussed two main approaches to active learning and provided algorithms for 
different active learning methods (Chapter~\ref{ch:al}). A simulation study 
indicated that $----$ would be the best learning algorithm for the 
financial application of the VS system. Similarly, we discussed various 
measures of graph distance and performed a simulation study to determine which 
methodology to implement in the VS system for the financial application 
(Chapter~\ref{ch:gc}). 

We then ran the VS system with $----$ in stage 1 and 
used $----$ to compute the graph comparison of the output $G$ with 
$G_i^{\text{num}}$ for all $i \in \{1,...,4\}$, the correlation graphs for 
Pearson, Spearman's, Kendall's, and distance correlation coefficient 
respectively. $----$ most closely matched the visual graph. We then utilized 
the stock selection methodology described in 
Section~\ref{sec:usage:stockselection} to select a portfolio of $k = 10$ stocks 
$P_i$ for all $i \in \{1,...,4\}$ based on $G_i^{\text{num}}$ based on data 
from year 1. Yearly returns were computed with the remainder of the 
observations in a simulation of the ``buy and hold'' strategy. As expected, the 
portfolio corresponding the correlation graph selected with the aid of the VS 
was one of the top performers (??).

The visualization tool presented in this work is an important step in 
streamlining the future of clean analysis. It provides a systematic way for 
confirming and/or suggesting dependencies among variables that match our visual 
concept of a dependence and produces an explicit decision tree that allow 
others to understand and replicate the data analysis process. This alleviates 
the problems associated with high-dimensional datasets and allows the 
user to quickly see ways in which the numerical model may have fallen short of 
the ``true'' relationship between variables. Nevertheless, there are several 
places to develop further work in order to refine the system and improve our 
concept of ``clean analysis.''