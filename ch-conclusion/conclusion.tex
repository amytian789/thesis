\section{Conclusion}
\label{sec:conclusion}

The proliferation of high-dimensional datasets has led analysts to rely on 
unverifiable numerical estimators. It would be useful to be able to construct 
high-dimensional visual correlation graphs as a way to verify numerical 
tests of dependence. Correlation graphs are important in portfolio selection as 
it is desirable to select stocks that are as uncorrelated as 
possible. However, high-dimensional visualization is problematic because there 
are too many potential plots to sort through manually; to be 
specific, there are ${n\choose 2} = n(n-1)/2$  plots where $n$ is the number of 
variables e.g. stocks that we are interested in. As a general solution to these 
issues, the visualization system actively learns user preferences regarding 
their notion of visual correlation, applies 
the fitted classifier to unlabeled data, provides visualization tools for the 
active learning output, and outputs the difference among the visual graph $G$ 
and some given numerical graph $G^{\text{num}}$ (Chapter~\ref{ch:visualizer}).

In this work, we focused specifically on the active learning and graph 
comparison components of the VS. We reviewed the two main approaches to active 
learning (efficient search through hypothesis space $\mathcal{H}$ and 
exploiting clustering tendencies in data) and provided algorithms for 
different active learning methods in Chapter~\ref{ch:al}. These methods include 
uncertainty sampling, query by committee (for which we proposed a revised 
framework), query by bagging, and min-max clustering. A simulation study 
with parameters that mimicked the intended qualities of the VS indicated that 
uncertainty sampling would be the ideal active learning method to be 
implemented in stage 1 of the VC for usage in 
the financial application of Chapter~\ref{ch:usage}. Similarly, we discussed 
various graph summarization difference metrics, proposed a method to compare 
the distance metrics of various graph pairs in order to select the most 
similar pair, and demonstrated the viability of the proposed procedure in 
Chapter~\ref{ch:gc}). The graph summarization difference and similarity 
selection are output functions of the VS that may be applied to any graphs and, 
subsequently, find a use in various other applications.

Our financial data contains the daily prices of 43 different healthcare 
stocks that were treated to get rid of time-dependency. The data is split in 
half with the first half (1998 to 2006) used as ``training'' input and the 
second half (2006 to 2014) used as ``testing'' output. Various correlation 
coefficients were applied to the training set to create $G^{i,\text{num}}$ for 
all $i \in \{1,...,4\}$, the numerical correlation graphs for Pearson's, 
Spearman's, Kendall's, and distance correlation coefficient respectively. We 
then ran the VS on the training set to determine the visual graph $G$. After 
computing the difference between each graph pair $(G^{i,\text{num}},G)$, the 
similarity selection procedure presented in 
Section~\ref{sec:gc:simulations} was used to find $G^*$, the numerical 
graph $G^{i,\text{num}}$ that most closely matches the visual graph $G$. The 
algorithm selected $G^{4,\text{num}}$, the distance correlation graph, as 
$G^*$. Then the stock selection methodology described in 
Section~\ref{sec:usage:stockselection} was used to select a portfolio $P^i$ 
of $k = 5$ stocks for each corresponding correlation graph $G^{i,\text{num}}$. 
The testing set was used to compute yearly returns for each portfolio in a 
simulation of the ``buy and hold'' strategy. $P^* = P^4$, the 
distance correlation portfolio, is the consistent top performer, demonstrating 
the reliability and usefulness of the VS in a concrete application. 
Furthermore, as can be seen 
in the results in Section~\ref{sec:usage:results}, the entire process is  
transparent, and results are reproducible given the recorded user responses 
to the VS queries. Transparency and reproducibility are 
other important purposes of the VS which can certainly be explored further.

The visualization system presented in this work is an important step towards
streamlining the future of clean analysis. It provides a systematic way for 
confirming and/or suggesting dependencies among variables that match the user's 
concept of visual dependence and produces an explicit decision tree that allow 
others to understand and replicate the data analysis process. The VS both 
alleviates the problems associated with high-dimensional data analysis and 
allows the user to quickly identify where the numerical model may have fallen 
short of the ``true'' relationship between variables. Nevertheless, there are 
several places to develop further work in order to refine the system and 
improve our concept of ``clean analysis.''