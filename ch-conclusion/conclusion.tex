%The importance of visualization.

Although this paper has focused on its financial applications, visualization systems nevertheless enable better decisions during any sort of data analysis. In the univariate case (Section \ref{sec:intro:me1}, \ref{sec:intro:me2}), it is clear that plotting has augmented the data analysis process by allowing the user to double-check their fitted models. In higher dimensions with more complex data sets and an inability to plot everything, it is even more believable that numerical methods cannot completely replace the valuable information obtained from visual methods. The visualization system developed in this text provides a way to safeguard data analysts against their own biases.

This visualization tool is one important step in streamlining the future of clean analysis. It provides a systematic way for confirming and suggesting dependencies among variables that match the analyst's concepts of a dependent (or ``interesting'' plot) and produces an explicit decision tree that allow others to understand and replicate the data analysis process. This removes the tediousness associated with high-dimensional data and allows the user to quickly see ways in which the numerical model may have fallen short of the ``true'' relationship between variables. In other words, it systematically provides the user a way to validate their methodology and model selection. Furthermore, it improves accountability in the analysis as it allows the decision-making process (in the form of a decision tree) to be clearer to those reviewing the results. Nevertheless, the graphical model that we develop is not fool proof, and further work can be done to refine the model and improve our concept of ``clean analysis.''
