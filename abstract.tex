Often in modern multivariate analyses, data analysts rely solely on statistical
estimators to explore the data. However, since each estimator inherently
performs well or poorly under different settings, data analysts are unable to
differentiate between the properties intrinsic to the dataset (the “truth”) and
the spurious properties the estimators add. Furthermore, the rapid increase in
computing resources has led to the proliferation of high-dimensional datasets,
which are tedious to visualize and analyze, making it difficult for analysts to
check their numerical results. In this paper, we contribute towards the notion
of “clean analysis” with the development of a visualization system for data
analysts to use in order to retain visual feedback when interacting with data. A
natural application of the system is in dependencies among financial equities,
which are often large, multivariate datasets. We wish to check a Gaussian
graphical model of 1000 historical observations of S\&P 500 stocks in order to
find independent relationships, a critical condition for re-balancing
portfolios. The visualization systems takes a numerical model and its associated
data as inputs. It then proceeds to extract features from plots of the response
against the explanatory variables and actively learns what visual patterns the
analyst finds promising. Afterwards, it automatically iterates through and
classifies thousands of possible plots. The system returns the plots it thinks
matches the analyst’s interests along with the decision tree. With this system,
analysts can combine visual feedback with numerical feedback from estimators to
make a more informed decision to proceed, modify, or scrap their existing
numerical model. Furthermore, the return of an explicit decision tree allows the
analyst to provide clear justification of their decisions and allows for
reproducible results, an important feature associated with increased
accountability. The visualization tool provides a simple and intuitive way to
improve predictive and portfolio management methods in the financial industry.
Moreover (and arguably more importantly), it increases standardization in the
data analysis process, thereby increasing accountability in an industry where
ambiguity can mean a global financial crisis.
