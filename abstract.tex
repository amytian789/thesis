Often in modern multivariate analyses, data analysts rely solely on statistical
estimators to explore the data. We are interested in visualizing 
high-dimensional correlation graphs as a way to verify numerical tests of 
dependence, which have far-reaching financial implications. High-dimensional 
visualization is problematic because (1) the number of plots to sort through 
increases quadratically as the number of variables increase, (2) it is tedious 
to verify numerical results with visual results and vice versa. We 
present a visualization tool that actively learns user preferences, applies 
the fitted classifier to unlabeled data, and outputs the difference among the 
numerical graph $G^{\text{num}}=(V,E^{\text{num}})$ and the visual graph 
$G=(V,E)$. 
As a specific response to the aforementioned problems, we focus on 
the active learning and graph comparison components of the visualization 
system. A simulation study with parameters that mimic the intended qualities of 
the system selected uncertainty sampling as the best procedure for use in the 
selection of healthcare stocks for a portfolio. Various graph summarization 
metrics are compiled to compute the difference between two graphs, and a 
procedure for selecting the most similar graph pairs given the differences is 
proposed and verified. 
Furthermore, we propose a simple but effective stock selection procedure to 
select a ``buy and hold'' portfolio of the $k$ stocks which are most 
uncorrelated with each other given a correlation graph, a proxy for 
independence. The healthcare stock price data is run through the visualization 
system, stocks are selected, and the portfolios' yearly returns are compiled. 
The results indicate that the portfolio 
corresponding to the numerical correlation graph which most resembles the 
visual correlation graph is the top performer. 
The visualization tool provides a simple and 
intuitive way to improve predictive and portfolio management methods in the 
financial industry. Moreover, it increases standardization in the data analysis 
process, thereby increasing accountability in an industry where ambiguity can 
mean a global financial crisis.
