Often in modern multivariate analysis, data analysts rely solely on statistical
estimators to explore the data. We are interested in using the notion of visual 
dependence to verify numerical tests of dependence (specifically, we 
focus on correlation metrics) and applying the results to portfolio selection, 
a setting that involves high-dimensional data sets. High-dimensional 
visualization is problematic because the number of pairwise plots to sort 
through increases quadratically as the number of variables increase.
We present a visualization system that actively learns the user's concept of
``visual correlation'', applies the resulting fitted classifier to unlabeled 
data to form a visual correlation graph $\hat{G}=(V,E)$, and outputs the 
difference between $\hat{G}$ and some given numerical correlation graph 
$\hat{G}^{\text{num}}$.
Specifically, we focus on the active learning and graph comparison components 
of the visualization system. We perform a simulation study with parameters that 
mimic the intended qualities of the system in order to select the best active 
learning method to use in the visualization system for the financial 
application. We compile various graph summarization metrics to compute the 
difference between two graphs (e.g. $\hat{G}$ and $\hat{G}^{\text{num}}$), and 
propose and verify a procedure for selecting $\hat{G}^*$, the numerical 
correlation graph most similar to the base graph $\hat{G}$. Furthermore, we 
propose a simple but effective stock selection 
procedure that, given a correlation graph,  selects a ``buy and hold'' 
portfolio of $k$ stocks which are as uncorrelated with each other as possible, 
a proxy for independence. 
Numerical correlation graphs $\hat{G}^{i, \text{num}}$ are formed from 
healthcare stock price data where $i \in I$ (the set of all correlation 
metrics), the data is fed into the visualization system to create $\hat{G}$, 
portfolios $P^i$ are selected from $\hat{G}^{i, \text{num}}$, and yearly 
returns are compiled. The results indicate that the portfolio $P^*$, which 
is selected from $\hat{G}^*$, is the top performer. Furthermore, all portfolios 
$P^i, i \in \{1,...,4\}$ outperform the S\&P 500, indicating that 
a more sophisticated selection strategy would yield even more fruitful returns. 
The VS may be further applied to improve upon other portfolio management 
techniques.