\chapter{Visualization System \label{ch:visualizer}}

Regardless of whether high-dimensional data visualization methods are
computationally heavy or interaction heavy, user interactivity is a critical
component of the analysis; it is simply a question of what 
degree~\cite{lius2016}. It is not enough to simply have user interaction, 
however. Given $n$ variables, there are a total of ${n \choose 2} = n(n-1)/2$ 
possible scatter plots of the data. This blows up quadratically as $n$ 
increases, which is infeasible for the analyst to sort through in any 
reasonable amount of time. Subsequently, automation is another necessary 
element in the task of visualizing high dimensional datasets.

We develop a system that first learns what visual patterns the data analyst 
finds promising, querying the user where the decision boundary is ambiguous. It 
then automatically iterates through all possible pairwise scatter plots and 
returns an adjacency matrix that captures the classification of each pair. 
Classifications are binary where the label 1 indicates ``visually correlated'' 
and the label 0 indicates ``not visually correlated''. The VS may then return 
the top ``visually correlated'' scatter plots for the user, perform line-up 
tests to refine the classifier (see Section~\ref{sec:futurework:lineup}), 
and/or compare the visual graph with some numerical graph(s) of the user's 
choice. This allows analysts to compare and contrast visual feedback with 
numerical algorithms for improved model selection. 
Figure~\ref{fig:visualizer:vs} is a visual summarization of the system.
       
\begin{figure}[htb]
	\begin{center}
		\includegraphics[width=1\linewidth]{ch-visualizer/figures/vs}
		\caption[Broad overview of the visualization system.]{Broad overview of 
		the visualization system.}
		\label{fig:visualizer:vs}
	\end{center}
\end{figure}

Section~\ref{sec:visualizer:scatterplot} describes characteristics of an 
interesting scatter plot; incorporating these ideas in the VS facilitates user 
accuracy in stage 1. 
Sections~\ref{sec:visualizer:al} and~\ref{sec:visualizer:plotgeneration} 
provide a brief overview of stages 1 and 2, respectively, and 
Section~\ref{sec:visualizer:plotgeneration:output} describes the VS output.
Section~\ref{sec:visualizer:focus} describes the focuses of the rest of this 
thesis, which respond to the problems and application posed in 
Chapter~\ref{ch:intro}.

\section{Scatterplot characterization}
\label{sec:visualizer:scatterplot}

\subsection{Characteristics of a ``good'' plot}
\label{sec:visualizer:scatterplot:goodplot}

The simplest scatterplot is the response against the observed variables. This,
however, may not be the best way to ascertain independence for the user. This
notion is illustrated in Figure~\ref{fig:visualizer:cdf}. The left plot appears
to be independent as it’s a cluster of points near the origin, but it's not
entirely clear due to the multitude of stray points around the concentrated
section. By looking at the outliers, it could also be argued that there is some
dependency. However, applying the CDF in both directions creates a plot
distributed on (0,1). It should be noted that this transformation is
non-destructive and preserves dependency in the data if it exists. The data is
clearly independent as the points appear to be uniformly distributed within the
plot.

\begin{figure}[htb]
	\begin{center}
		\includegraphics[width=0.75\linewidth]{ch-visualizer/figures/cdf}
		\caption[A plot of $y$ against $x$ after the CDF is applied in both
		directions.]{A plot of $y$ against $x$ with no transformation (left) 
		and after
			the CDF is applied in both directions (right). The code for this 
			example may be
			found in ~\ref{sec:appendicies:cdf}}
		\label{fig:visualizer:cdf}
	\end{center}
\end{figure}

Restricting the plot to a unit box allows analyst's visual systems to focus on
locations where there is low spatial frequency, which is ideal for detecting
dependence~\cite{hofert2016}. The effects of this can be progressively observed
by looking from the left to the right in
Figure~\ref{fig:visualizer:hofertoldford} below.

\begin{figure}[htb]
	\begin{center}
		\includegraphics[width=0.75\linewidth]{ch-visualizer/figures/hofertoldford}
		\caption[Scatterplots of independent $U(0,1)$ random variables and the
		pseudo-observation pairs $(U_{t,j},U_{t,j+1}),j\in 
		\{1,2,3\}$.]{Scatterplots of
			(a) independent $U(0,1)$ random variables and (b,c,d) the 
			pseudo-observation
			pairs $(U_{t,j},U_{t,j+1}),j\in \{1,2,3\}$. Ticker abbreviations: 
			AAP = Advanced
			Auto Parts, AMZN = Amazon.com Inc, AN = AutoNation Inc., AZO = 
			AutoZone Inc.
			Images from Hofert and Oldford 2016~\cite{hofert2016}}
		\label{fig:visualizer:hofertoldford}
	\end{center}
\end{figure}

\subsection{Feature extraction from plot}
\label{sec:visualizer:features}

\subsubsection{Numerical features}

((Our goal is to quantify various features of a scatter plot. One category of
features are numerical features. These include Pearson correlation, tests of
independence, mutual information criterion, etc))

\subsubsection{Visual features}

((The other category of feature are visual. How many points are near the center
of the plot? How many points lie above the linear regression line? ))



\section{Active learning (Stage 1)}
\label{sec:visualizer:al}

The main goal of stage 1 is to learn the user's interests. This requires the 
system to select (``query'') data (which are $n$ characterizations of $d\choose 
2$ plots in the case of the VS as described in 
Section~\ref{sec:visualizer:scatterplot:features}) for the analyst (the 
``oracle'') to label (classify). This is \textbf{stage 1}. 
The learner may then utilize 
a classification model (discriminant analysis, naive bayes, decision tree(s), 
logistic regression, etc.) that trains on labeled data to ``learn'' user 
interests. The user's interests are encoded in a classifier (some instance of 
the classification model) that is applied to automatically label the rest of 
the data (For more on the 
semantic differences between ``classification model'' and ``classifier'' in 
this body of work, see Figure~\ref{fig:visualizer:al:tree}). This is 
\textbf{stage 2} (Section~\ref{sec:visualizer:plotgeneration}). 
As such, it is important to make the process as efficient as possible to avoid 
redundancy for the end user. There are various methods that may be used for 
querying in stage 1~\cite{dasgupta2011}:

\tablespacing
\begin{itemize}
	\item \textbf{Supervised learner}: This learner queries a single, random 
	subset of all unlabeled data. It ignores the rest of the data when refining 
	the classifier
	\item \textbf{Semisupervised learner}: Similar to a supervised learner, a 
	semisupervised learner queries a single, random subset of all 
	unlabeled data but proceeds to utilize the remaining unlabeled data to 
	better inform the final classifier
	\item \textbf{Active learner}: An active learner selects its queries in a 
	non-random, intelligent manner to reduce the hypothesis space $\mathcal{H}$ 
	of all possible classifiers that may explain the data.
\end{itemize}
\bodyspacing

It has been 
shown that when a learning algorithm is allowed to choose its next query, it 
performs better with less training; as such, we choose to utilize active 
learning to select the plots to be queried by the oracle in stage 
1~\cite{settles2010}. Chapter~\ref{ch:al} goes into detail on different active 
learning methodologies as this section is primarily focused on its role in the 
system.

\begin{figure}[htb]
	\begin{center}
		\includegraphics[width=0.75\linewidth]{ch-visualizer/figures/tree}
		\caption[Classifiers and classification models]{Although both figures 
		on the left and right are slightly different classifiers, they 
		are both instances of (extremely simple) decision trees, a type of 
		classification model. Other models include discriminant analysis, naive 
		bayes, random forest, logistic regression, etc.}
		\label{fig:visualizer:al:tree}
	\end{center}
\end{figure}

\subsection{Initialization of active learner}
\label{sec:visualizer:al:initialization}

It is problematic to start from scratch; how does the system determine
the best first point of ambiguity when it knows nothing (the hypothesis space is
everything)? A classic method is to simply select $k$ random data instances for 
the user to label. As initialization is not the focus of this work, the VS 
currently utilizes this methodology.

Alternatively, we can exploit the fact that the
user is already providing a numerical model that they believe to be a good
representation of the data which they would like the visualization system to
check visually. Given this data, the system builds a decision tree that utilizes
the various properties of the plots to determine whether one is interesting or
uninteresting. Doing so greatly narrows the hypothesis space and makes it easier
to determine points of ambiguity. However, to reconcile with the fact that the
user wishes to check the numerical model and may not necessarily believe it is a
good representation of fit, the learner must check whether the initial decision
tree is a proper fit (This may be achieved with line-up tests, which are 
described briefly in Section ~\ref{sec:futurework:lineup}). As the user then 
proceeds to label various conditional
plots as ``interesting'' or ``not interesting,'' the classifier learns the
user’s interest and continues to evolve. This models plot characteristics that
the user found interesting to study.

\subsection{Query selection}
\label{sec:visualizer:al:tree}

Post-initialization, the active learner cleverly queries vital plots so that 
the system can best learn the user’s interests. The system first determines 
which features it is uncertain about classifying and then returns a plot 
matching those characteristics to the user. This allows the system to utilize 
its classification model of choice to build a better classifier more 
efficiently. \textbf{
	It is important to distinguish between the active learner, which 
	selects the next queries from the pool of unlabeled data and \textit{may 
		use 
		its own classification model(s) to aid in query selection}, and the VS, 
	which 
	uses a \textit{single classification model} to fit both the initialization 
	and actively selected queries in order to build a model to classify the 
	user's 
	interests and label the remaining plots in Stage 2 }
(Section~\ref{sec:visualizer:plotgeneration}). 
Various active learning (selection) algorithms include uncertainty sampling, 
query by committee, query by bagging, and min-max clustering, all of which are 
described more thoroughly in Chapter~\ref{ch:al}.
\section{Automated plot generation (Stage 2)}
\label{sec:visualizer:plotgeneration}

\subsection{User interaction with active learning output}
\label{sec:visualizer:plotgeneration:user}

((Now that the system has learned the user’s interest, the user should be able to understand his/her own interests. We provide visualization tools of the resulting classifier itself such as a heat map, which there are also multiple ways to visualize))

\subsection{Plot generation and feedback}
\label{sec:visualizer:plotgeneration:feedback}

((Equipped with the learned classifier of the user’s interest, the system can now automatically generate thousands of new plots and label them automatically. The most interesting plots are returned the user along with the explanatory variable (including possible interaction terms) corresponding to the plot))
\section{Specific focus: active learning and graph comparison}
\label{sec:visualizer:focus}

The active learning component is the bread and butter of the visualization 
system and forms stage 1 of the system. It solves for the 
tedious nature of classifying ${n \choose 2}=n(n-1)/2$ pairwise scatter plots 
in high-dimensional visualization where $n$ is the number of variables (e.g. 
stocks). 
The next issue to consider is the usage of the visual results. This is a 
question of the VS output, and graph comparison is a solution to this problem 
that is both useful in selecting the numerical method most similar to the 
visual graph (which we are interested in conjunction with numerical correlation 
graphs for portfolio selection) and informative to the analyst. Furthermore, in 
instances where the analyst is using the VS to check their numerical models, 
large differences encourage the user to investigate whether their selected 
method was truly appropriate for the dataset at hand. 
Subsequently, 
we primarily focus on active learning methods in Chapter~\ref{ch:al} and graph 
comparison methods in Chapter~\ref{ch:gc}. These methods are added to the 
current iteration of the VS, which is used (with healthcare stock data) 
in conjugation with numerical correlation graphs to perform stock selection in 
Chapter~\ref{ch:usage}. The current iteration of the VS is under development in 
the \texttt{graphicalModelEDA} package by \texttt{linnylin92}~\cite{lin2017}.

It should be noted that the VS is a large project which is still under 
construction, and there is much room for 
refinement beyond the work that we do in this thesis, which develops two of its 
major components. Future extensions of the VS may be found in 
Section~\ref{sec:futurework}; there are several ideas on improving various 
aspects of the system which are not the focus of this thesis.
