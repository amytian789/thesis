\section{Specific focus: AL and GC}
\label{sec:visualizer:focus}

The active learning component is the bread and butter of the visualization 
system and forms the first stage of the system. Furthermore, it solves for the 
tedious nature of classifying $d \choose 2$ plots in high-dimensional 
visualization , which was first brought up in Section~\ref{sec:intro:problem}. 
The second issue with high-dimensional visualization is the verification of the 
numerical results against the visual result. This is a question of the VS 
output, and graph comparison is a solution to this problem that is both 
informative to the analyst and useful in selecting the most relevant numerical 
method where large differences encourage the analyst to investigate whether 
their selected numerical method was truly appropriate for the dataset at hand. 
As such, the graph comparison output is most useful (as opposed to plot 
generation and line-up tests) in the case of numerical correlation 
graphs, which we are interested in for their financial applications. As such, 
the primary focus of this work that follows are active learning methods 
(Chapter~\ref{ch:al}) and graph comparison methods (Chapter~\ref{ch:gc}). These 
methods are applied to the current iteration of the VS, which is then utilized 
in conjugation with numerical correlation graphs to perform stock selection.

It should be noted that the VS is a large project and may certainly be further 
refined despite the work that we do to refine two major components of the 
system. Ideas for future extensions of the VS may be found in 
Section~\ref{sec:futurework}. There are several recommendations for 
improvements that can be made to different aspects of the system which are not 
the focus on this work.