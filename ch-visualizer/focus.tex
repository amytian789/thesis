\section{Specific focus: active learning and graph comparison}
\label{sec:visualizer:focus}

The active learning component is the bread and butter of the visualization 
system and forms stage 1 of the system. It solves for the 
tedious nature of classifying ${n \choose 2}=n(n-1)/2$ pairwise scatter plots 
in high-dimensional visualization where $n$ is the number of variables (e.g. 
stocks). 
The next issue to consider is the usage of the visual results. This is a 
question of the VS output, and graph comparison is a solution to this problem 
that is both useful in selecting the numerical method most similar to the 
visual graph (which we are interested in conjunction with numerical correlation 
graphs for portfolio selection) and informative to the analyst. Furthermore, in 
instances where the analyst is using the VS to check their numerical models, 
large differences encourage the user to investigate whether their selected 
method was truly appropriate for the dataset at hand. 
Subsequently, 
we primarily focus on active learning methods in Chapter~\ref{ch:al} and graph 
comparison methods in Chapter~\ref{ch:gc}. These methods are added to the 
current iteration of the VS, which is used (with healthcare stock data) 
in conjugation with numerical correlation graphs to perform stock selection in 
Chapter~\ref{ch:usage}. The current iteration of the VS is under development in 
the \texttt{graphicalModelEDA} package by \texttt{linnylin92}~\cite{lin2017}.

It should be noted that the VS is a large project which is still under 
construction, and there is much room for 
refinement beyond the work that we do in this thesis, which develops two of its 
major components. Future extensions of the VS may be found in 
Section~\ref{sec:futurework}; there are several ideas on improving various 
aspects of the system which are not the focus of this thesis.