\section{Literature review}
\label{sec:al:litreview}

It is important to consider the situation in which the learning algorithm 
receives the points in which it selects a query from. There are three different 
scenarios in which the learner may request queries as described by 
Settles~\cite{settles2010}:
 
\tablespacing
\begin{enumerate}
	\item \textbf{Membership query synthesis}: The learner may select a query 
	from an unlabeled samples in $X$.
	\item \textbf{Stream-based selective sampling}: An unlabeled sample is 
	randomly selected from $X$, and the learner decides whether to query or not.
	\item \textbf{Pool-based sampling}: $k$ unlabeled samples are randomly 
	selected from $X$, and the learner picks one to query.
\end{enumerate}
\bodyspacing

\noindent While the methods above may apply to many different active learning 
environments, the specific situation for the VS is pool-based selective 
sampling. Because it is computationally expensive to extract features from 
every single $d\choose2$ plot 
(Section~\ref{sec:visualizer:scatterplot:features}), membership query synthesis 
is not an option. While stream-based selective sampling may work, it has a 
similar problem as membership query synthesis because there are no constrains
on the number of samples that the algorithm may choose to discard. 

The next problem, then, is to determine the informativeness of unlabeled 
instances that are presented by the methodologies above. This is the crux of 
active learning as it allows for intelligent selection of queries.
Dasgupta identifies two different approaches to active learning that are 
fundamental drivers to the process of query selection~\cite{dasgupta2011}: 

\tablespacing
\begin{enumerate}
	\item \textbf{Efficient search through hypothesis space $\mathcal{H}$}: 
	The idea is to select a query that shrinks $\mathcal{H}_t$, the set of all 
	possible classifiers at time $t$ that explain the labeled data, as much as 
	possible. 
	\item \textbf{Exploiting cluster structure in data}: 
	The idea is to cluster the data and select queries based on cluster 
	structure (i.e. query from each cluster). While clusters may be split over 
	time if they are discovered to be non-homogenous, a learner may leverage 
	situations where clusters are fairly homogeneous in order to classify 
	points by propagating the labels to its neighbors. 
\end{enumerate}
\bodyspacing

\noindent Uncertainty sampling and query by committee are active learning 
algorithms that are a form of the aforementioned efficient search through the 
hypothesis space. These algorithms may be found in 
Sections~\ref{sec:al:methods:uncertainty} to ~\ref{sec:al:methods:bagging} 
respectively. We also present a clustering partitioning algorithm in 
Section~\ref{sec:al:methods:clustering} that seeks to exploit the cluster 
structure in the data.
