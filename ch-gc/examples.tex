\section{Similarity selection}
\label{sec:gc:examples}

We propose a method to select graph $\hat{G^i}, i \in \{1,...,k\}$ where 
$\hat{G^i}$ is most similar to the ``base'' graph $\hat{G}$. $\hat{G^i}$ is 
selected out of $k$ given graphs each associated with a vector 
$\overrightarrow{d^{i,j}}$ where $i\neq j$ and $i,j \in \{1,...,k\}$ of their 
differences with each difference $\overrightarrow{d^{i,j}_m}$ for $m \in 
\{1,...,6\}$ corresponding to one of the 6 summarization methods listed earlier 
in Section~\ref{sec:gc:methods}. And demonstrate its 
viability with a simulation study. This method also eliminates the need to 
normalize the resulting difference metrics, which will not be on the same scale 
within $\overrightarrow{d_{i,j}}$, as it compares each metric to its own metric 
rather than computes an aggregate ``difference'' between graphs $\hat{G}^i$ and 
$\hat{G}^j$.
This method is applied in Chapter~\ref{ch:usage} to compare the visual graph 
(output from the VS) and the various correlation graphs (with correlation 
coefficients explained in Section~\ref{sec:intro:correlation})



Plus code...(for simulation study)



Extension:
Our normalization has bounded the values between 0 and 1, but that doesn't mean 
they're uniformly distributed between 0 and 1. 
(see sheet on this method)