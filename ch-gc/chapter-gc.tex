\chapter{Graph comparison \label{ch:gc}}

Graph comparison may be reduced to a problem of measuring the difference 
between two graphs. 
A primary goal of the visualization system is to allow the user to better 
verify their numerical results with visual intuition. As such, there are two 
main goals for graph comparison output: (1) allow the user to gain some 
intuition about the qualities of the visual graph in order to better understand 
their perceptions of a ``visual trend'' in the data, and (2) provide a metric 
that quantifies the difference between the visual graph and numeric graph 
(supplied by the user). With the knowledge of both in hand, the user may 
reevaluate their numeric methods and utilize the results in alternate 
applications (For example, the second metric is utilized to select stocks for a 
portfolio in Chapter~\ref{ch:usage}). It should be noted that we operate under 
the assumption that the graphs are labeled. This avoids the sticky issue of 
isomorphism and more naturally lends itself to the visualization system's 
goals. As the VS is concerned with practical application in data 
analytics, each variable in a dataset should already correspond to some label 
(e.g. various stocks).
What follows is broad overview of two graph comparison methods in 
Section~\ref{sec:gc:litreview} with more details on the graph summarization 
comparison method in Section~\ref{sec:gc:methods}. Finally, we propose a method 
to select the graph $\hat{G^i}, i \in \{1,...,k\}$ out of $k$ given graphs 
where $\hat{G^i}$ is most similar to the ``base'' graph $\hat{G}$ and perform 
a simulation study to demonstrate the method's viability in 
Section~\ref{sec:gc:examples}.

\section{Graph summarization}
\label{sec:gc:litreview}

There are two ways of thinking about graph comparison. Let 
$\hat{G^1}=(V^1,E^1)$ and $\hat{G^2}=(V^2,E^2)$ each be some labeled graph. 

\tablespacing
\begin{enumerate}
	\item \textbf{Graph summarization:} Compute some vector 
	$v^1$ on $\hat{G^1}$ that ``summarizes'' various properties that
	it has. Similarly, compute $v^2$ on $\hat{G^2}$ to capture 
	the same properties that $v^1$ does. Finally, compute 
	$f(v^1,v^2)$ where $f$ is some distance 
	function (e.g. Euclidean distance). A high value indicates strong 
	dissimilarity while a low value indicates similarity. 
	
	\item \textbf{Graph distance:} Create a valid distance function $f$ on the 
	graphs $\hat{G^1}, \hat{G^2}$ directly rather than a vector of metrics 
	($v^1,v^2$). 
	Compute $f(\hat{G^1},\hat{G^2})$. Again, a high value indicates strong 
	dissimilarity while a low value indicates similarity. The \textit{random 
	walk kernel} is one such example of a graph distance 
	metric~\cite{vishwanathan2010}.
\end{enumerate}
\bodyspacing

We specifically focus on the graph summarization comparison because it 
fulfills both desired criteria for the visualization system; an understanding 
of each summarization method allows the user to gain intuition on the qualities 
of the graph, and the difference between two graph summarization methods can be 
quantified easily. To be more specific, we seek to better understand the 
qualities of existing graph summarization methods and their associated 
advantages and disadvantages. 
Understanding these metrics is important due to the complex nature of graphs.
It is natural to ask why we cannot just plot the graphs instead (which are 
typically in the form of either an adjacency matrix or a list of edges) and 
examine the result for visual patterns in order to understand the qualities 
of the graph. This is unfeasible for two reasons (see 
Figure~\ref{fig:gc:arr_density}):

\tablespacing
\begin{itemize}
	\item \textbf{Arrangement:} Different arrangements of nodes on the visual 
	plane may have a profound effect on the interpretability of the graph; from 
	the start, it is unclear what the best arrangement might be, and it is 
	unfeasible for an analyst to try all possible arrangements 
	especially as the number of variables increase.
	
	\item \textbf{Density:} The more nodes there are, the more edges there may 
	be, and the more dense a graph may become, making it incredibly difficult 
	to interpret. Furthermore, it makes it more difficult to find anomalies 
	among the variable's relationships (represented by edges). A non-existent 
	edge in a dense graph is just as important as an existent edge in a sparse 
	graph because each indicates a relationship which isn't quite like all the 
	others.
\end{itemize}
\bodyspacing

\begin{figure}[htb]
	\begin{center}
		\includegraphics[width=1\linewidth]{ch-gc/figures/arr_density}
		\caption[Difficulties with graph visualization.]{
		Suppose we have a graph $G=(V,E)$ with clusters $C^1,C^2$ and $|V|=18$. 
		Let $C^1$ be composed of nodes $(V_1,...,V_6)$, and $C^2$ be composed 
		of nodes $(V_7,...,V_{18})$. 
		\textit{Left:} $C^1$, the smaller cluster, has been placed inside 
		$C^2$, the larger cluster. It is difficult to discern any meaningful 
		patterns; each node appears to be connected to every other node as the 
		density of the cluster $C^2$ makes it almost impossible to see that of 
		$C^1$.
		\textit{Right:} The nodes have been rearranged in a manner that clearly 
		distinguishes between $C^1$ and $C^2$. Now it is evident 
		that $C^1$ is missing $E_{3,6}$. It is still difficult to tell, 
		however, that 
		$C^2$ is even missing an edge $(E_{13,17})$ in the first place.
		$C^1$ is more sparse, which makes it easier to visually perceive 
		anomalies in the graph.}
		\label{fig:gc:arr_density}
	\end{center}
\end{figure}

Graph summarization acts as a proxy for visually exploring the graph 
itself. Each metric, which is associated with a different characteristic of the 
graph, can be parsed and later combined to 
``reconstruct'' or better understand the qualities of each graph, subsequently 
allowing the user to better understand \textit{exactly where} the visual and 
numerical graphs differ (should the final graph comparison metric suggest that 
the two graphs are highly different). Graph distance functions skip that 
critical step by going through 
the graphs directly.

To put it in another way, graph summarization has the \textit{exact opposite 
problem} that is present in data analytics as it is practiced today (discussed 
in Chapter~\ref{ch:intro}, this is the motivation behind the VS and this 
thesis). 
The visualization system allows the user to use the visual qualities of pairwise
scatter plots to better understand numerical qualities of the data, but graph 
summarization allows the user to use numerical qualities to better understand 
the visual qualities of a graph. 

With an understanding of the properties of various graph summarization methods, 
the graph summarization difference metric may be more informative. In the VS, 
the graph summarization difference computes a list of graph summarization 
metrics $m^i$ for $\hat{G}^i$ and $m$ for the base graph $\hat{G}$ 
in order to compute a 
single vector $d$ that quantifies the difference between them with 
some difference function (e.g. Euclidean distance, L2, etc.). 
As such, a natural byproduct of the computation is the pair
$(m^i,m)$, which (armed 
with an understanding of the values) provides insight on the qualities of the 
visual graph and numerical graph \textit{on an individual level} and where 
exactly their differences and similarities lie.
\section{Overview of graph summarization methods}
\label{sec:gc:methods}

There are many different ways to summarize a graph. However, akin to how a
correlation coefficient cannot sufficiently capture all aspects of dependency 
on its own, there is no ``best'' summarization method that can summarize 
every quality of a graph on its own. As such, this section discusses various 
graph summarization methods, the qualities that they measure, and their 
drawbacks. Furthermore, it is useful to normalize the returned 
summarization difference in order to understand the value properly, so a 
normalization method is proposed for each metric difference. 
In section~\ref{sec:gc:simulations}, we propose a method to select the most 
similar graph pair given a vector of differences where each difference 
corresponding to one of the six summarization methods listed below.

Most methods listed below are implemented in the \texttt{igraph} package in 
\texttt{R}. As such, we do not implement most methods ourselves as in 
Chapter~\ref{ch:al}. The code for 
all methods may be found in Appendix~\ref{sec:appendicies:gc:engine}. It should 
be noted that there are various distance functions that could be used. 
Euclidean distance, L1, and L2 are currently implemented (The code for the 
distance functions may also found in Appendix~\ref{sec:appendicies:gc:engine}). 

For all methods below, let $\hat{G}=(V,E)$ be some labeled graph.

\subsection{Centrality}

The centrality of a graph $\hat{G}$ is given by $\mathcal{C}$ and measures 
of the importance of its 
nodes. Let $\overrightarrow{c}$ be a vector of length $V$ where each element 
$\overrightarrow{c_i}$ is the ``importance'' of node $i$. Then the centrality 
of $\hat{G}$ is given by
$$c = \sum\limits^{len(V)}_{i=1} \max_{\forall j \in \{1,...,len(V)\}} 
\overrightarrow{c_j} - \overrightarrow{c_i}$$
The importance of the node may be measured by utilizing metrics such as 

\tablespacing
\begin{itemize}
	\item \textbf{Degree:} The degree of a node is the number of edges the node 
	is part of. In other words, it is the number of direct paths from the node 
	to all other nodes. The higher the degree, the more important the node 
	is.
	\item \textbf{Closeness:} The closeness of a node is the average 
	shortest path length (the path which traverses the smallest number of 
	edges) between the node and all other nodes. The higher the closeness, 
	the closer the node is to all other nodes (in terms of path length), the 
	more important the node is.
	\item \textbf{Betweenness:} The betweenness of a node is the number of 
	times the node is part of the shortest path between two other nodes. Direct 
	paths are excluded. The higher the betweenness, the more important the node 
	is as it provides a critical cost-saving link (path length) for the path 
	between many other nodes. 
\end{itemize}
\bodyspacing

The normalized centrality of a graph $\hat{G}$ may be computed as 	
$\frac{Cen(\hat{G})}{\max Cen(\hat{G})}$ where $\max Cen(\hat{G})$ is the 
maximum centrality that graph $\hat{G}$ may have (for whatever metric is 
specified). Taking the difference between two normalized scalars will also 
yield a normalized scalar.
Degree centrality is implemented in \texttt{centr\_degree}, 
closeness centrality is implemented in \texttt{centr\_clo}, and betweenness 
centrality is implemented in \texttt{centr\_betw} from the \texttt{igraph} 
package. By default, each function normalizes $\mathcal{C}$.

Now consider a graph whose edges are rotated as in 
Figure~\ref{fig:gc:drawbacks_rotation}; the centrality of the overall graph 
will not change as the edges are in the same relative positions, but the graph 
itself is clearly different. The computed centrality difference (with distance 
function L2 and for all centrality types) between the graphs pictured in 
Figure~\ref{fig:gc:drawbacks_rotation} is 0, which suggests they are exactly 
same! This is due to the ``global'' nature of the centrality metric, which 
ignores the individual label of the nodes. Thus, when edges are rotated and 
nodes change, the graph ``looks'' exactly the same.
Should centrality be examined at the level of a node and not a graph, 
this issue would be accounted for (though normalization is not as trivial). It 
is more difficult to parse a node-level metric especially given many variables, 
so this paper does not delve into detail or implement this metric.

\subsection{Assortativity}

Assortativity measures the level of resemblance among the connected nodes of a 
graph $\hat{G}$ based on vertex category labels. For graphs with uncategorized 
nodes (as in the examples of this chapter and the 
application in Chapter~\ref{ch:usage}), it is common to use $deg(V_i)-1$
as the category label for node $V_i$ instead. For 
an edge $E_{i,j}$, the degree of nodes $V_i$ and $V_j$ are collected in two 
distinct vectors $\overrightarrow{v^1}, \overrightarrow{v^2}$. Then the 
assortativity metric of a graph $\hat{G}$ is $\mathcal{A} = 
\rho(\overrightarrow{v^1},\overrightarrow{v^2})$ where $\rho$ is the Pearson 
correlation as defined in Section~\ref{sec:intro:correlation}. As such, 
the assortativity of $\hat{G}$ is summarized in a single value 
$-1 \leq \mathcal{A} \leq 1$ where the interpretation is the same as that of 
the Pearson correlation coefficient. For instance, $\mathcal{A}=1$ when nodes 
with high/low degrees in $\hat{G}$ are mostly connected to other nodes with 
high/low degrees in $\hat{G}$, respectively. In other words, if 
$\mathcal{A}$ is large, then connected vertices tend to have the same 
qualities, and when $\mathcal{A}$ is small, then connected vertices tend to 
have opposite qualities. Since assortativity involves a correlation 
computation, the result is already standardized. Again, taking the difference 
between two normalized scalars will also yield a normalized scalar.
Assortativity is called with the \texttt{assortativity\_degree} function from 
the \texttt{igraph} package.

Similarly, the assortativity difference metric is susceptible to rotated edges 
as shown in Figure~\ref{fig:gc:drawbacks_rotation}. Assortativity is another 
``global'' metric that measures the correlation among the degree of nodes but 
does not care about the exact nodes themselves. As such, the metric can be 
potentially misleading in situations where the actual nodes matter. In fact, 
the computed difference (with distance function L2) between the graphs in 
Figure~\ref{fig:gc:drawbacks_rotation} is 0.0032, which suggests the graphs are 
very similar when they are (visually) different.







\begin{figure}[htb]
	\begin{center}
		\includegraphics[width=1\linewidth]{ch-gc/figures/drawbacks_rotation}
		\caption[A graph pair with rotated edges.]{A graph pair with rotated 
			edges. Graph summarization differences (L2 difference function): 
			
			\begin{tabular*}{.857 \textwidth}{ | c | c | c | }
				\hline
				Centrality (degree) & Centrality (closeness) & Centrality 
				(betweenness) \\
				\hline
				0 & 0 & 0 \\
				\hline
			\end{tabular*}
			\begin{tabular*}{0.984 \textwidth}{ | c | c | c | }
				\hline
				Community (random walk) & Community (infomap) & Community 
				(betweenness) \\
				\hline 
				0.9677 & 0.9677 & 0.9677 \\
				\hline
			\end{tabular*}
			\begin{tabular*}{0.947 \textwidth}{ | c | c | c | c | }
				\hline
				Assortativity & Distance matrix & Edge connectivity & Edge 
				density 
				histogram \\
				\hline
				0 & 0.3557 & 0 & 0 \\ 
				\hline
			\end{tabular*}
		}
		\label{fig:gc:drawbacks_rotation}
	\end{center}
\end{figure}





\subsection{Community}

The community of a graph $\hat{G}$ is given by $\mathcal{S}$, a set of its 
dense subgraphs. Subgraphs may be detected with any of the following algorithms:

\tablespacing
\begin{itemize}
	\item \textbf{Random walks:} By implementing a random walk within a graph 
	to determine the various subgraphs (or, in other words, to ``describe'' the 
	community), the random walk procedure takes advantage of the natural idea 
	that random walks within a graph will tend to stay in the same subgraph. 
	This occurs because a subgraph is often densely connected within but 
	sparsely connected to other subgraphs, so there is an increased probability 
	of proceeding via an edge that is within the same subgraph.
	This approach to building a community is called with the 
	\texttt{cluster\_walktrap} function from the \texttt{igraph} package.	
	
	\item \textbf{Infomap:} An infomap utilizes random walks to build a 
	community by minimizing the expected length that it takes a random walk 
	to ``describe'' the community.
	This approach to building a community is called with the 
	\texttt{cluster\_infomap} function from the \texttt{igraph} package.
	
	\item \textbf{Betweenness:} As described earlier, the betweenness of a node 
	is the number of times the node is part of the shortest path between two 
	other nodes (where direct paths are excluded). The algorithm sequentially 
	removes the node with the highest betweenness and places it on a tree, 
	recalculating betweenness of the existing graph every time a removal is 
	performed. This algorithm once again takes advantage of the natural idea 
	that subgraphs tend to be dense among themselves but with few edges between 
	them. As such, edges which connect separate subgraphs should have high 
	betweenness because all shortest paths from one subgraph to another must 
	pass through them. 
	This approach to building a community is called with the 
	\texttt{cluster\_edge\_betweenness} function from the \texttt{igraph} 
	package.
\end{itemize}
\bodyspacing

There is no real way or need to normalize $\mathcal{S}$ as it is not a numeric 
value. What is important to note, however, is that comparison among 
graphs $\hat{G^1}=(V^1,E^2)$ and $\hat{G^2}=(V^2,E^2)$ via the community graph 
summarization metric (which \textit{does} return a numeric value)
must compare $\mathcal{S}^1$ and $\mathcal{S}^2$ with a method such as the 
Jaccard distance. Consider sets labeled $\mathcal{S}^1=(1,1,2,2,3,3)$ and 
$\mathcal{S}^2=(4,4,5,5,6,6)$. These are the exact same set labellings, but 
they are numbered differently; it is clear, then, that a traditional distance 
metric such as Euclidean distance will not capture the relationship properly. 
The Jaccard distance compares the actual sets rather than their literal values. 
The metric is already normalized between 0 and 1.
The Jaccard distance is obtained by subtracting the Jaccard similarity, which 
is computed by the \texttt{cluster\_similarity} function in the 
\texttt{clusteval} package, from 1.

Where community has trouble is when a graph pair has many sparse clusters. 
Consider the simplified example found in Figure~\ref{fig:gc:drawbacks_comm}. 
While one graph has two physical clusters and the other has three, they are not 
all that visually different because only one edge has been removed. The 
community difference values are 0, 0.2727, and 0.2727 for random walk, infomap, 
and betweenness respectively. This increase is extremely significant when 
considering that the ``best case'' scenario where the community 
(6,7,8,9,10,11,12)  from Figure~\ref{fig:gc:drawbacks_comm} (left) is 
completely split into (6,7,8,9) and (10,11,12) 
Figure~\ref{fig:gc:drawbacks_comm} (right) yields a Jaccard difference of 
0.3870968. Interestingly, the random walk community difference is 0; this is 
likely because it is a \textit{single random walk} which leaves more room for 
errors and may not fully explore the clustering structure especially if 
clusters are completely isolated as is the case in both graphs of the set 
illustrated in Figure~\ref{fig:gc:drawbacks_comm}.










\begin{figure}[htb]
	\begin{center}
		\includegraphics[width=1\linewidth]{ch-gc/figures/drawbacks_comm}
		\caption[A graph pair with two and three clusters, respectively.]{A 
			graph pair with two and three clusters, respectively. Only one edge 
			has been removed, so the graphs do not appear that extremely 
			different. Graph summarization differences (L2 difference function):
			
			\begin{tabular*}{.857 \textwidth}{ | c | c | c | }
				\hline
				Centrality (degree) & Centrality (closeness) & Centrality 
				(betweenness) \\
				\hline
				0.0002 & 0.0002 & 0.0102 \\
				\hline
			\end{tabular*}
			\begin{tabular*}{0.984 \textwidth}{ | c | c | c | }
				\hline
				Community (random walk) & Community (infomap) & Community 
				(betweenness) \\
				\hline 
				0 & 0.2727 & 0.2727 \\
				\hline
			\end{tabular*}
			\begin{tabular*}{0.947 \textwidth}{ | c | c | c | c | }
				\hline
				Assortativity & Distance matrix & Edge connectivity & Edge 
				density 
				histogram \\
				\hline
				0.0075 & 0.2233 & 0 & 0.0139 \\ 
				\hline
			\end{tabular*}
		}
		\label{fig:gc:drawbacks_comm}
	\end{center}
\end{figure}









\subsection{Distance matrix}

The distance matrix $M$ of a graph $\hat{G}$ is a square matrix that 
contains the shortest path length 
between all pairs of variables. Each element $M_{i,j}$ 
contains the shortest path distance between node $i$ to node $j$. If no path 
exists, it is conventional to set the value to infinity. However, since the 
main purpose is to compare two matrices, setting a value of infinity doesn't 
make sense. Alternatively, the value may be set to zero since we define the 
shortest path distance earlier to be \textit{the path which traverses the 
smallest number of edges}, so a \textit{direct path} between nodes has path 
length 1. Thus, the shortest path length can never be 0, which allows the 
existence of a ``non-existent path'' to be distinguished from the ``length of 
an existing path''. It is important to note that the matrix itself is repeated; 
the upper and lower triangles are the same. Furthermore, the entries across the 
diagonal must be 0. Thus, the matrix can be unfurled into a vector 
$\overrightarrow{m}$ with the values of all $\binom{|V|}{2}$ unique pairs.
It is less trivial to normalize the distance matrix difference metric. It would 
be extremely incorrect, for instance, to normalize by dividing the raw distance 
by $z = dist\big((|V|-1),0\big) * len(\overrightarrow{m})$ because it is 
impossible for a graph to be constructed such that every node's 
\textit{shortest path} to all other nodes is exactly length $|V|-1$ (the 
maximum possible degree) or for a graph to be constructed such that the value 
is even close to $z$. 
The simplest method would be to treat the difference between an empty graph (a 
graph with no edges) and a graph with a single Hamiltonian path (a path that 
visits each vertex once) as the upper bound. This bound is actually slightly 
above 1, but it is close enough for the purposes of interpretation. 
The distance matrix method is called with the \texttt{distances} function from 
the \texttt{igraph} package. \texttt{distances} returns infinity when a path 
doesn't exist, so the output is adjusted accordingly.

The distance-matrix is clearly node-specific as the metric relies on the 
shortest paths between all unique pairs of nodes. As such, edge rotation is not 
a problem, but clustering becomes an issue when there are few clusters. 
Consider Figure~\ref{fig:gc:drawbacks_cluster}; one graph has two physical 
clusters and the other doesn't (but appears to be clustered visually). 
The removal of edge (4,5) forces the shortest path between node pairs involving 
one from each cluster to drop to 0, having a strong effect on the distance 
matrix distance since everything is now interconnected. 
The actual distance matrix difference for the graph pair pictured in 
Figure~\ref{fig:gc:drawbacks_cluster} is 0.7708, which suggests that the graphs 
in the pair are very different when they are actually quite similar visually as 
the mind perceives the same clustering structure. 
Although Figure~\ref{fig:gc:drawbacks_comm} is similar to 
Figure~\ref{fig:gc:drawbacks_cluster} in that both involve cluster 
manipulation, the larger number of clusters in 
Figure~\ref{fig:gc:drawbacks_comm} ensures that disconnecting two clusters will 
not change the distance matrix difference as much; less of the graph is 
interconnected, so fewer values will change to 0. In fact, the distance matrix 
difference of Figure~\ref{fig:gc:drawbacks_comm} happens to be 0.2233, which is 
much lower than the distance matrix difference of 
Figure~\ref{fig:gc:drawbacks_cluster}. 









\begin{figure}[htb]
	\begin{center}
		\includegraphics[width=1\linewidth]{ch-gc/figures/drawbacks_cluster}
		\caption[A graph pair with one (no) and two clusters, respectively.]{A 
			graph pair with one (no) and two clusters, respectively. Only one 
			edge has been removed, so the graphs do not appear that extremely 
			different. Graph summarization differences (L2 difference function):
			
			\begin{tabular*}{.857 \textwidth}{ | c | c | c | }
				\hline
				Centrality (degree) & Centrality (closeness) & Centrality 
				(betweenness) \\
				\hline
				0.0002 & 0.0899 & 0.1933 \\
				\hline
			\end{tabular*}
			\begin{tabular*}{0.984 \textwidth}{ | c | c | c | }
				\hline
				Community (random walk) & Community (infomap) & Community 
				(betweenness) \\
				\hline 
				0 & 0 & 0 \\
				\hline
			\end{tabular*}
			\begin{tabular*}{0.947 \textwidth}{ | c | c | c | c | }
				\hline
				Assortativity & Distance matrix & Edge connectivity & Edge 
				density 
				histogram \\
				\hline
				0.0032 & 0.7708 & 0.25 & 0.0139 \\ 
				\hline
			\end{tabular*}
		}
		\label{fig:gc:drawbacks_cluster}
	\end{center}
\end{figure}










\subsection{Edge connectivity}

The edge connectivity of a graph $\hat{G}$ is given by $\mathcal{E}$, the 
number of edges that need to be deleted in order to \textit{disconnect} 
$\hat{G}$ such that there exists two nodes $V_i,V_j$ in $\hat{G}$ which do not 
have a path between them.
%More specifically, it is the number of edges that need to be 
%deleted so that $\hat{G}$ is partitioned into $k$ distinct subgraphs and is
%no longer ``strongly'' connected. 
As the metric is upper bounded by the minimum 
degree of any node within a graph, the metric may be normalized by computing 
$\frac{\mathcal{E}}{\min_i deg(V_i)}$ for all possible nodes $i$. Then, the 
difference normalization will naturally follow.
The edge connectivity of a graph is computed 
with the \texttt{edge\_connectivity} function from the \texttt{igraph} package. 

It is natural to see where this metric could fall through the cracks as it also 
is unaffected by the labels of each node. Consider 
a graph whose edges are rotated as in Figure~\ref{fig:gc:drawbacks_rotation}; 
the number of edges needed to disconnect the graph won't change as the edges 
are in the same relative positions, but the graph itself is clearly different. 
In fact, the computed edge connectivity difference (with distance function L2) 
between the graphs pictured in Figure~\ref{fig:gc:drawbacks_rotation} is 0, 
which suggests they are exactly same!

\subsection{Edge density histogram}

Edge density captures the number of nodes in a graph $\hat{G}$ that have a 
specified number of degrees. Naturally, a larger number of high-degree nodes is 
a signal for density while a larger number of low-degree nodes is a signal for 
sparsity. For $\hat{G}$, the edge density is summarized in a 
vector $\overrightarrow{v}$ of length $b$, the number of desired ``bins'' 
(e.g. bin 1 is $1-5$ degrees, bin 2 is $6-10$ degrees, etc.). Then the value 
$\overrightarrow{v_i}$ is the number of nodes which correspond to bin $i$. This 
metric may be normalized by computing $\frac{\overrightarrow{v_k}}{V}$ for 
all $k \in \{1,...,b\}$ since there are a total of $V$ nodes which also allows 
the difference computation to be normalized. 
This summarization method is not implemented in the \texttt{igraph} package. 

The edge density histogram difference metric is also susceptible to rotated 
edges as illustrated in Figure~\ref{fig:gc:drawbacks_rotation} because it 
aggregates various metrics in bins, which discards more detailed information 
about each individual node; the number of high and low-degree nodes 
will not change even if the edges are rotated, but the graph is still clearly 
different. In fact, the computed edge density histogram difference (with 
distance function L2) between the exact graphs in 
Figure~\ref{fig:gc:drawbacks_rotation} is 0, which suggests the graphs are the 
exact same!
\section{Examples}
\label{sec:gc:examples}

Table summarization of method outputs when applied to various examples we've 
thought of previously